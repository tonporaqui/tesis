% \section{Conclusiones}\label{Conclusiones}
% Esta investigación ha marcado un camino innovador en el análisis del rendimiento estudiantil, gracias en gran parte a la introducción de la metodología ACAMD. A través de la combinación sinérgica de SHAP y DoWhy, ACAMD no solo nos ha permitido identificar las variables clave, sino también desentrañar las relaciones causales subyacentes, ofreciendo insights que trascienden las correlaciones superficiales y proporcionan una comprensión más rica y matizada de los factores que influyen en el rendimiento académico.

% La metodología ACAMD se sitúa en la vanguardia de la investigación educativa, ofreciendo un enfoque que integra la interpretación de modelos y la inferencia causal. Su aplicación tiene el potencial de revolucionar nuestra comprensión de los factores educativos y cómo estos interactúan en formas complejas y a menudo inesperadas, abriendo nuevas avenidas para la investigación y la práctica en nuestro campo.

% Con la metodología ACAMD como piedra angular de nuestro análisis, hemos logrado abordar los objetivos de investigación planteados al inicio de este trabajo. Cada fase de nuestra investigación fue informada y guiada por ACAMD, demostrando su eficacia en proporcionar un análisis profundo y causal que es esencial para formular estrategias educativas informadas y efectivas.

% Nuestros hallazgos revelan la importancia de variables como \texttt{hito1}, \texttt{exitosos}, \texttt{fallidos}, \texttt{e42} y \texttt{e29} en el contexto académico. Sin embargo, lo que hace que esta investigación sea particularmente reveladora no es solo la identificación de estas variables, sino la comprensión de cómo interactúan y afectan el rendimiento de los estudiantes. La capacidad de SHAP para descomponer la importancia predictiva de cada variable ha sido crucial, proporcionando insights que van más allá de la mera correlación.

% Por su parte, DoWhy nos ha permitido explorar el mundo de la causalidad, aportando una nueva dimensión a nuestro análisis. Más allá de saber qué variables son importantes, ahora comprendemos mejor \textit{por qué} son importantes y cómo pueden influir activamente en el rendimiento académico. El análisis con DoWhy ha revelado relaciones causales significativas entre la resolución de la guía y el rendimiento académico, proporcionando insights valiosos para la toma de decisiones basada en evidencia causal.

% Con base en nuestros descubrimientos, se sugiere que futuras investigaciones en este ámbito consideren proporcionar relaciones más detalladas entre variables como \texttt{hito1}, \texttt{hito2} así como otros hitos que se esperen evaluar y las preguntas específicas de la guía que pertenecen a estos hitos. Además, sería beneficioso incorporar información sobre la identificación de estados de ánimo de los estudiantes. Técnicas modernas propuestas por organizaciones líderes como Google y OpenAI han demostrado ser efectivas en el análisis de estados de ánimo y detección de emociones \cite{google_mood_tracking_2022, google_sentiment_api_2022, openai_sentiment_neuron_2022, openai_sentiment_analysis_2022, openai_gpt_emotion_2022, openai_chatgpt_emotion_2023}. Estas técnicas, basadas en el análisis de lenguaje natural y aprendizaje profundo, pueden ofrecer insights valiosos sobre cómo el estado emocional de un estudiante puede influir en su rendimiento académico.

% En conclusión, esta investigación no solo ha arrojado luz sobre las dinámicas del rendimiento estudiantil, sino que también ha establecido un camino para futuras investigaciones. Mediante el uso combinado de análisis predictivo y causal, podemos aspirar a desarrollar estrategias educativas más informadas y efectivas, siempre con el objetivo de mejorar la experiencia y el éxito académico de los estudiantes.


\section{Conclusión}

A lo largo de esta investigación, se ha demostrado la eficacia y relevancia de la metodología ACAMD en el análisis de datos educativos. Esta metodología, que integra técnicas de comparación, interpretación y causalidad, ha permitido obtener insights profundos sobre la influencia de la guía de programación en el rendimiento académico de los estudiantes.

Los hallazgos derivados de SHAP y DoWhy, en particular, han arrojado luz sobre la significativa relación entre la resolución de ciertas preguntas de la guía y el éxito académico en la primera evaluación del curso. Estas herramientas, al ser aplicadas conjuntamente, ofrecen perspectivas distintas pero complementarias, permitiendo una comprensión holística de las variables en juego.

Es crucial destacar que, aunque los datos pueden contener sesgos, los resultados son consistentes en relación con la significancia de las preguntas. La variable e18, por ejemplo, adquiere una relevancia especial en el contexto educativo, ya que se resuelve en clases con la asistencia del profesor. Esta interacción directa no solo refuerza el conocimiento del alumno, sino que también le proporciona una base sólida para enfrentar futuros desafíos académicos.

Sin embargo, es importante reconocer la complejidad inherente al análisis de datos en el ámbito educativo. Factores como el estado de ánimo del estudiante al momento de presentarse a la prueba, su salud física, el ambiente familiar, entre otros, pueden influir en el rendimiento y no siempre se capturan en los datos. Por lo tanto, mientras que los resultados de esta investigación proporcionan insights valiosos, siempre deben interpretarse considerando estas limitaciones.

Con base en nuestros descubrimientos, se sugiere que futuras investigaciones en este ámbito consideren proporcionar relaciones más detalladas entre variables como \texttt{hito1}, \texttt{hito2} así como otros hitos que se esperen evaluar y las preguntas específicas de la guía que pertenecen a estos hitos. Además, sería beneficioso incorporar información sobre la identificación de estados de ánimo de los estudiantes. Técnicas modernas propuestas por organizaciones líderes como Google y OpenAI han demostrado ser efectivas en el análisis de estados de ánimo y detección de emociones \cite{google_mood_tracking_2022, google_sentiment_api_2022, openai_sentiment_neuron_2022, openai_sentiment_analysis_2022, openai_gpt_emotion_2022, openai_chatgpt_emotion_2023}. Estas técnicas, basadas en el análisis de lenguaje natural y aprendizaje profundo, pueden ofrecer insights valiosos sobre cómo el estado emocional de un estudiante puede influir en su rendimiento académico.

En resumen, la metodología ACAMD ha demostrado ser una herramienta poderosa para el análisis de datos educativos, permitiendo no solo identificar variables clave, sino también entender las relaciones causales entre ellas. Se espera que los hallazgos de esta investigación sirvan como base para futuros estudios en el área y para la formulación de estrategias educativas basadas en evidencia.
