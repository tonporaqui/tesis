\hypertarget{Conclusiones}{%
    \section{Conclusiones}\label{Conclusiones}}

A lo largo de esta investigación, hemos emprendido un viaje analítico en el que las herramientas modernas, como SHAP y DoWhy, se han convertido en aliados esenciales para desentrañar las complejidades del rendimiento estudiantil. La naturaleza multidimensional del rendimiento académico exige un enfoque que no solo identifique correlaciones, sino que también desentrañe las relaciones causales que subyacen en estos patrones.

Un aspecto crucial de nuestra investigación fue la comparación exhaustiva de diferentes algoritmos, tanto de clasificación como de regresión. Esta comparación nos permitió adentrarnos en el proceso de selección del mejor modelo para llevar a cabo nuestras predicciones con SHAP. Gracias a este análisis detallado, pudimos identificar y seleccionar el algoritmo óptimo que no solo proporcionó precisión en las predicciones, sino que también ofreció una mayor interpretabilidad al analizar la importancia de las características.

Nuestros hallazgos revelan la importancia de variables como \texttt{hito1}, \texttt{exitosos}, \texttt{fallidos}, \texttt{e42} y \texttt{e29} en el contexto académico. No obstante, lo que hace que esta investigación sea particularmente reveladora no es solo la identificación de estas variables, sino la comprensión de cómo interactúan y afectan el rendimiento de los estudiantes. La capacidad de SHAP para descomponer la importancia predictiva de cada variable ha sido crucial, proporcionando insights que van más allá de la mera correlación.

Por su parte, DoWhy nos ha permitido explorar el mundo de la causalidad, aportando una nueva dimensión a nuestro análisis. Más allá de saber qué variables son importantes, ahora comprendemos mejor \textit{por qué} son importantes y cómo pueden influir activamente en el rendimiento académico. El análisis con DoWhy ha revelado relaciones causales significativas entre la resolución de la guía y el rendimiento académico, proporcionando insights valiosos para la toma de decisiones basada en evidencia causal.

Con base en nuestros descubrimientos, se sugiere que futuras investigaciones en este ámbito consideren proporcionar relaciones más detalladas entre variables como \texttt{hito1}, \texttt{hito2} así como otros hitos que se esperen evaluar y las preguntas específicas de la guía que pertenecen a estos hitos. Además, sería beneficioso incorporar información sobre la identificación de estados de ánimo de los estudiantes. Técnicas modernas propuestas por organizaciones líderes como Google y OpenAI han demostrado ser efectivas en el análisis de estados de ánimo y detección de emociones \cite{google_mood_tracking_2022, google_sentiment_api_2022, openai_sentiment_neuron_2022, openai_sentiment_analysis_2022, openai_gpt_emotion_2022, openai_chatgpt_emotion_2023}. Estas técnicas, basadas en el análisis de lenguaje natural y aprendizaje profundo, pueden ofrecer insights valiosos sobre cómo el estado emocional de un estudiante puede influir en su rendimiento académico. Incorporar información adicional como la asistencia a clases, el sexo, la edad, entre otras, podría enriquecer significativamente el análisis. Específicamente, herramientas como SHAP podrían utilizarse para visualizar el impacto de estas variables adicionales y entender su relación con la probabilidad de aprobación.


Finalmente, es esencial reconocer que, aunque nuestros hallazgos son robustos y prometedores, aún existen factores externos no controlados que pueden influir en la respuesta de un alumno a una guía o en su rendimiento general. La incorporación de variables adicionales y la continuación de este tipo de análisis permitirán reducir la incertidumbre y mejorar la precisión de nuestras predicciones.

En este viaje investigativo, hemos tenido la oportunidad única de crear una metodología novedosa que integra las capacidades de SHAP y DoWhy. Esta metodología, que se encuentra en la intersección de la interpretación de modelos y la inferencia causal, nos ha permitido no solo identificar las variables clave que influyen en el rendimiento académico, sino también entender las relaciones causales que existen entre ellas. La integración de estas dos herramientas ha demostrado ser una estrategia poderosa, permitiéndonos ir más allá de las técnicas tradicionales de análisis de datos y ofreciendo una comprensión más profunda y matizada de los factores que afectan el rendimiento estudiantil. Creemos que esta metodología tiene el potencial de revolucionar la forma en que se abordan los estudios en el ámbito educativo, proporcionando una hoja de ruta para investigaciones futuras que aspiren a una comprensión más completa y causal de los fenómenos estudiados.

En conclusión, esta investigación no solo ha arrojado luz sobre las dinámicas del rendimiento estudiantil, sino que también ha establecido un camino para futuras investigaciones. Mediante el uso combinado de análisis predictivo y causal, podemos aspirar a desarrollar estrategias educativas más informadas y efectivas, siempre con el objetivo de mejorar la experiencia y el éxito académico de los estudiantes.
