\hypertarget{Conclusiones}{%
    \section{Conclusiones}\label{Conclusiones}}

En esta investigación, hemos navegado por un camino analítico, utilizando herramientas sofisticadas como SHAP y DoWhy, y hemos introducido una metodología innovadora, \textit{Análisis Causal Asistido por Minería de Datos} (ACAMD). Esta metodología nos ha permitido profundizar en las complejidades del rendimiento académico, yendo más allá de las correlaciones superficiales para entender las relaciones causales subyacentes.

La rigurosa selección de modelos nos llevó a elegir el RandomForestClassifier, destacando por su precisión predictiva y claridad interpretativa. A través de SHAP, identificamos variables cruciales como \texttt{hito1}, \texttt{exitosos}, \texttt{fallidos}, \texttt{e42}, y \texttt{e29}. Sin embargo, la verdadera innovación surgió al aplicar DoWhy dentro del marco de ACAMD, lo que nos permitió explorar las causas fundamentales detrás de estas variables y cómo influencian el rendimiento estudiantil.

Con base en nuestros descubrimientos, se sugiere que futuras investigaciones en este ámbito consideren proporcionar relaciones más detalladas entre variables como \texttt{hito1}, \texttt{hito2} así como otros hitos que se esperen evaluar y las preguntas específicas de la guía que pertenecen a estos hitos. Además, sería beneficioso incorporar información sobre la identificación de estados de ánimo de los estudiantes. Técnicas modernas propuestas por organizaciones líderes como Google y OpenAI han demostrado ser efectivas en el análisis de estados de ánimo y detección de emociones \cite{google_mood_tracking_2022, google_sentiment_api_2022, openai_sentiment_neuron_2022, openai_sentiment_analysis_2022, openai_gpt_emotion_2022, openai_chatgpt_emotion_2023}. Estas técnicas, basadas en el análisis de lenguaje natural y aprendizaje profundo, pueden ofrecer insights valiosos sobre cómo el estado emocional de un estudiante puede influir en su rendimiento académico.

La metodología ACAMD, en la intersección de la interpretación de modelos y la inferencia causal, ha abierto nuevas puertas en nuestra comprensión, permitiéndonos identificar no solo las variables clave, sino también las relaciones causales entre ellas. Esta estrategia integrada promete revolucionar los estudios educativos, proporcionando una hoja de ruta para futuras investigaciones que buscan una comprensión más profunda y causal.

En conclusión, este estudio ha arrojado luz sobre las dinámicas complejas del rendimiento estudiantil y ha trazado un camino para investigaciones futuras. Con el análisis predictivo y causal combinado a través de ACAMD, estamos mejor posicionados para desarrollar estrategias educativas informadas y efectivas, con el objetivo primordial de mejorar la experiencia educativa y el éxito académico de los estudiantes.
