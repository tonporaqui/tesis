\hypertarget{objetivo_especifico}{%
    \section{Objetivo Especifico}\label{Objetivo Especifico}}

Con base en el problema planteado, se propone llevar a cabo una investigación detallada para analizar la relación entre la resolución de una guía de programación y el éxito académico en el ramo de introducción a la programación de la Universidad Andrés Bello. La propuesta de investigación se estructura de la siguiente manera:

\begin{enumerate}
    \item Recopilación de datos: Se obtendrá el conjunto de datos "dataset a 2021", que contiene información sobre los resultados de la guía y el rendimiento en la solemne 1 del año 2021.

    \item Revisión bibliográfica: Se llevará a cabo una revisión exhaustiva de la literatura académica relevante relacionada con el tema. Se analizarán estudios que aborden la relación entre la resolución de guías de programación y el éxito académico, así como investigaciones sobre el uso de técnicas de SHAP y XAI en contextos educativos.


    \item Análisis descriptivo: Se llevará a cabo un análisis descriptivo de los datos recopilados, incluyendo medidas de tendencia central y dispersión, para examinar la distribución de los resultados de la guía y la solemne 1. Se realizarán comparaciones entre grupos de estudiantes para identificar posibles correlaciones.

    \item Construcción del modelo predictivo: Con los datos recopilados, se desarrollará un modelo predictivo para evaluar la influencia de la resolución de la guía en el rendimiento académico y su capacidad predictiva de deserción. Se explorará la causalidad utilizando herramientas como DoWhy, y se aplicarán técnicas de SHAP y XAI para interpretaciones claras del modelo.

    \item Validación del modelo: Se realizarán pruebas y validaciones del modelo predictivo utilizando técnicas como validación cruzada y análisis de sensibilidad. Esto garantizará la confiabilidad y robustez de las predicciones generadas por el modelo, así como la calidad de las explicaciones proporcionadas por las técnicas de SHAP y XAI.

    \item Análisis e interpretación de resultados: Se analizarán los resultados, considerando la influencia de la guía en el rendimiento académico y su capacidad predictiva de deserción. Se utilizarán las explicaciones de SHAP, XAI y los resultados de inferencia causal de DoWhy para comprender las variables más relevantes en las predicciones.

    \item Conclusiones y recomendaciones: Se presentarán las conclusiones derivadas de la investigación, destacando las principales contribuciones y hallazgos. Además, se ofrecerán recomendaciones prácticas para la universidad en cuanto al uso de la guía de programación como recurso educativo y su impacto en el rendimiento académico de los estudiantes, utilizando las explicaciones interpretables generadas por las técnicas de SHAP y XAI.

\end{enumerate}

La realización de esta propuesta de investigación permitirá obtener una comprensión más profunda de la relación entre la resolución de una guía de programación y el éxito académico en el ramo de introducción a la programación. Además, proporcionará información valiosa para mejorar las estrategias de apoyo a los estudiantes y tomar decisiones informadas en relación con la implementación y promoción de la guía como recurso educativo en la universidad, utilizando técnicas de SHAP y XAI para obtener explicaciones interpretables de los modelos predictivos.

