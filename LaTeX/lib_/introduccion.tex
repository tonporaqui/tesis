\hypertarget{Introducción}{%
\section{Introducción}\label{Introducción}}

La programación se ha consolidado como una habilidad esencial en el mercado laboral actual, siendo un pilar fundamental para el avance tecnológico. Dada su importancia, numerosas universidades han implementado cursos introductorios de programación con el objetivo de preparar adecuadamente a los futuros profesionales. No obstante, dada la naturaleza compleja de la programación y la falta de experiencia previa de muchos estudiantes, no es raro enfrentar dificultades en estos cursos. Para mitigar este problema, algunas instituciones ofrecen guías de programación que contienen ejercicios y problemas prácticos. Surge entonces una interrogante relevante: ¿Está la resolución de estas guías directamente relacionada con el éxito académico en el curso de introducción a la programación?

En este contexto, el presente trabajo aborda esta cuestión centrado en la experiencia de la Universidad Andrés Bello. Se propone el uso del marco unificado SHAP (Shapley Additive exPlanations) y XAI para garantizar la confianza en los modelos de AI/ML en el ámbito educativo, con el fin de predecir la relevancia de la resolución de guías para el éxito en el curso y su potencial predictivo en relación con la deserción estudiantil.

Más allá de las técnicas convencionales de análisis, la inferencia causal se presenta como una herramienta prometedora para desentrañar y predecir fenómenos complejos, como el rendimiento académico. La fusión del aprendizaje profundo con la inferencia causal, tal como se detalla en el trabajo de Geffner et al. \cite{geffner2022deep}, promete brindar perspectivas más detalladas sobre las relaciones causales inherentes en los conjuntos de datos.