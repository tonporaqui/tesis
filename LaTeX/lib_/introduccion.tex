\hypertarget{Introducción}{%
\section{Introducción}\label{Introducción}}

La programación se ha consolidado como una habilidad esencial en el mercado laboral actual, siendo un pilar fundamental para el avance tecnológico. Dada su importancia, numerosas universidades han implementado cursos introductorios de programación con el objetivo de preparar adecuadamente a los futuros profesionales. No obstante, dada la naturaleza compleja de la programación y la falta de experiencia previa de muchos estudiantes, no es raro enfrentar dificultades en estos cursos.

A nivel internacional, la programación y su enseñanza han sido objeto de múltiples investigaciones. \textit{Fincher et al.}~\cite{fincher2006predictors} exploraron los predictores de éxito en un primer curso de programación, identificando factores clave que pueden influir en el rendimiento de los estudiantes. Por su parte, \textit{Leeper y Silver}~\cite{leeper1982predicting} analizaron cómo ciertas habilidades cognitivas y la experiencia previa pueden predecir el éxito en la programación. En un enfoque más moderno, \textit{Watson et al.}~\cite{watson2013predicting} investigaron cómo el comportamiento de programación de los estudiantes, registrado y analizado, puede prever su desempeño en el curso. \textit{Azcona y Smeaton}~\cite{azcona2017targeting} se centraron en identificar a los estudiantes en riesgo utilizando predictores de compromiso y esfuerzo en un curso introductorio de programación. Finalmente, \textit{Erodogan et al.}~\cite{erodogan2008exploring} exploraron los predictores psicológicos del logro en programación, destacando la importancia de factores como la autoeficacia y la motivación.

En el contexto chileno, la enseñanza de la programación también ha sido objeto de atención. \textit{Jones Eduardo et al.}~\cite{metodologias2022activas} discutieron las metodologías activas para enseñar programación a estudiantes de ingeniería civil informática, destacando la importancia de un enfoque práctico y centrado en el estudiante. \textit{Miguel Jara Gómez et al.}~\cite{oportunidades2017integrar} exploraron las oportunidades y desafíos de integrar la enseñanza de la programación en el currículo escolar chileno. El \textit{Instituto de Estadística, Universidad de Chile}~\cite{campamento2023programacion} informó sobre un campamento de programación competitiva, resaltando la creciente importancia de la programación en la educación superior. \textit{GONZÁLEZ Mariela et al.}~\cite{programa2022acompanamiento} presentaron un programa de acompañamiento académico para estudiantes de primer año, subrayando la necesidad de apoyar a los estudiantes en sus primeros pasos en la universidad. Por último, \textit{Rodrigo Fábrega Lacoa et al.}~\cite{ensenanza2016lenguajes} discutieron la enseñanza de lenguajes de programación en la escuela, destacando su relevancia en la formación temprana de los estudiantes.

Dada la importancia de la programación y su enseñanza, surge una interrogante relevante: ¿Está la resolución de guías de programación directamente relacionada con el éxito académico en el curso de introducción a la programación?

En este contexto, el presente trabajo aborda esta cuestión centrado en la experiencia de la Universidad Andrés Bello. Se propone el uso del marco unificado SHAP (Shapley Additive exPlanations) y XAI para garantizar la confianza en los modelos de AI/ML en el ámbito educativo, con el fin de predecir la relevancia de la resolución de guías para el éxito en el curso y su potencial predictivo en relación con la deserción estudiantil.

Más allá de las técnicas convencionales de análisis, la inferencia causal se presenta como una herramienta prometedora para desentrañar y predecir fenómenos complejos, como el rendimiento académico. La fusión del aprendizaje profundo con la inferencia causal, tal como se detalla en el trabajo de Geffner et al. \cite{geffner2022deep}, promete brindar perspectivas más detalladas sobre las relaciones causales inherentes en los conjuntos de datos.
