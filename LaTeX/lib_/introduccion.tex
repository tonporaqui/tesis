\hypertarget{Introducción}{%
    \section{Introducción}\label{Introducción}}
    \vfill
La programación es una habilidad cada vez más demandada en el mercado laboral
 y es fundamental para el desarrollo de la tecnología. 
Por esta razón, muchas universidades ofrecen cursos de introducción a la 
programación para formar a los futuros profesionales en esta área. 
Sin embargo, algunos estudiantes pueden tener dificultades para aprobar estos 
cursos debido a la complejidad de los conceptos y la falta de experiencia 
previa en programación. 
Para ayudar a los estudiantes a enfrentar estos desafíos, algunas 
universidades ofrecen guías de programación que contienen ejercicios y 
problemas para resolver. 
En este contexto, surge la pregunta: 
¿la resolución de una guía de programación está relacionada con el éxito 
académico en el ramo de introducción a la programación? 
Este trabajo examina esta cuestión en el contexto de la 
Universidad Andrés Bello y propone el uso de redes bayesianas para generar 
predicciones sobre su relevancia para aprobar el ramo y su capacidad 
predictiva para la deserción. 
Los resultados pueden ser útiles para mejorar las estrategias de apoyo a 
los estudiantes y la toma de decisiones en relación con las guías como 
recurso educativo.
\vfill