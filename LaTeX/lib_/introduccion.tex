\hypertarget{Introducción}{%
\section{Introducción}\label{Introducción}}

La programación, en el contexto contemporáneo, no es solo una habilidad técnica, sino una competencia esencial para navegar y contribuir en un mundo cada vez más digitalizado. Las universidades, reconociendo su importancia, han integrado cursos introductorios de programación en sus currículos. Sin embargo, la naturaleza intrínsecamente desafiante de la programación, combinada con la falta de experiencia previa de muchos estudiantes, a menudo resulta en dificultades en el aprendizaje.

A nivel global, la enseñanza y aprendizaje de la programación ha sido objeto de extensas investigaciones. \textit{Fincher et al.}~\cite{fincher2006predictors} llevaron a cabo un estudio detallado sobre los predictores de éxito en un primer curso de programación, identificando variables clave que pueden influir en el rendimiento académico. En una línea similar, \textit{Leeper y Silver}~\cite{leeper1982predicting} examinaron la relación entre habilidades cognitivas específicas, experiencia previa y éxito en programación. \textit{Watson et al.}~\cite{watson2013predicting} adoptaron un enfoque más contemporáneo, analizando cómo el comportamiento de programación registrado de los estudiantes puede ser un indicador de su desempeño futuro. \textit{Azcona y Smeaton}~\cite{azcona2017targeting} se centraron en la identificación temprana de estudiantes en riesgo, utilizando predictores de compromiso y esfuerzo. Por otro lado, \textit{Erodogan et al.}~\cite{erodogan2008exploring} se adentraron en el ámbito psicológico, investigando cómo factores como la autoeficacia y la motivación pueden influir en el logro en programación.

En Chile, la enseñanza de la programación ha recibido una atención especial. \textit{Jones Eduardo et al.}~\cite{metodologias2022activas} abordaron las metodologías activas en la enseñanza de programación, subrayando la necesidad de un aprendizaje práctico y centrado en el estudiante. \textit{Miguel Jara Gómez et al.}~\cite{oportunidades2017integrar} discutieron las oportunidades y desafíos de integrar la programación en el currículo escolar chileno. El informe del \textit{Instituto de Estadística, Universidad de Chile}~\cite{campamento2023programacion} sobre un campamento de programación competitiva resalta la creciente importancia de la programación en la educación superior. \textit{GONZÁLEZ Mariela et al.}~\cite{programa2022acompanamiento} presentaron un programa de apoyo académico para estudiantes de primer año, destacando la importancia del soporte en las etapas iniciales de la educación universitaria. Además, \textit{Rodrigo Fábrega Lacoa et al.}~\cite{ensenanza2016lenguajes} reflexionaron sobre la enseñanza de lenguajes de programación en la educación básica, enfatizando su relevancia en la formación temprana.

Ante este panorama, surge una pregunta esencial: ¿Está la resolución de guías de programación directamente relacionada con el éxito académico en el curso introductorio de programación?

Este trabajo se centra en la experiencia de la Universidad Andrés Bello y propone el uso del marco unificado SHAP (Shapley Additive exPlanations) y XAI para garantizar la confianza en los modelos de AI/ML en el ámbito educativo. El objetivo es determinar la relevancia de la resolución de guías para el éxito en el curso y su potencial predictivo en relación con la deserción estudiantil.

La inferencia causal, más allá de las técnicas convencionales de análisis, emerge como una herramienta valiosa para comprender y predecir fenómenos complejos, como el rendimiento académico. La combinación de aprendizaje profundo con inferencia causal, como se detalla en el trabajo de \textit{Geffner et al.}~\cite{geffner2022deep}, promete ofrecer insights más profundos sobre las relaciones causales presentes en los conjuntos de datos.
