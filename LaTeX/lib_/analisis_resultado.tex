\hypertarget{analisis_resultado}{%
    \section{Análisis de resultados}\label{Análisis de resultados}}

Esta sección emplea la metodología \textit{Análisis Causal Asistido por Minería de Datos} (ACAMD) para profundizar en el impacto de resolver las guías de programación sobre el éxito académico de los estudiantes en el curso introductorio de programación. ACAMD, que incorpora tanto SHAP como DoWhy, permite un análisis exhaustivo y causal a partir del conjunto de datos \texttt{dataset} del año 2021.

El análisis se centra en \texttt{sol1}, identificada como nuestra variable clave. Utilizando SHAP, evaluamos la influencia de \texttt{sol1} en el rendimiento académico, seguido de la construcción de un modelo causal con DoWhy. Este enfoque responde a nuestra pregunta central de investigación: ¿Cómo se relaciona la resolución de una guía de programación con el éxito académico en el curso introductorio de programación en la Universidad Andrés Bello?

A través de ACAMD, no solo validamos la relevancia de \texttt{sol1} sino que también descubrimos las relaciones causales potenciales entre la participación activa de los estudiantes en las guías de programación y sus logros académicos. Estos hallazgos son fundamentales para el desarrollo de estrategias pedagógicas efectivas y la optimización de las guías como herramientas educativas.



\subsection{Análisis de datos}

Una vez recopilados los datos, se realizará un análisis descriptivo para examinar la distribución de los resultados en la guía de programación y la primera evaluación. Además, se llevará a cabo un análisis de correlación entre las variables mencionadas anteriormente para identificar posibles relaciones y patrones significativos.

\subsection{Descripción de la base de datos}

A continuación, se presenta la descricion de la base de datos.

\begin{table}[H]
    \centering
    \caption{Descripción de variables}
    \begin{tabular}{|l|p{0.6\linewidth}|}
        \hline
        \textbf{Variable}        & \textbf{Descripción}   \\
        \hline
        sol1                     & Indica la calificación obtenida en la primera evaluación, las calificacion parte de la nota 0 hasta la maxima 7 \\
        \hline
        exitosos                 & Indica la cantidad de preguntas respondidas correctamente en la guía, a medida que se van respondiendo correctamente esta va generando un resumen de de total respondidas exitosamente \\
        \hline
        fallidos                 & Indica la cantidad de intentos realizados para resolver preguntas de la guia \\
        \hline
        hito1                    & Son las espectativas a cumplir del aprendizaje del curso, se construye por medio de conjuntos de preguntas. \\
        \hline
        hito2                    & Son las espectativas a cumplir del aprendizaje del curso, se construye por medio de conjuntos de preguntas. \\
        \hline
        programa                 & Indica al programa de estudio. \\
        \hline
        Columnas e0 hasta la e52 & Son representacion de los resultados de las preguntas de la guia este es binario, 1 equivale a respondida exitosamente y cero no respondida.  \\
        \hline
    \end{tabular}
    \label{tab:variables}
\end{table}

Estas columnas son relevantes para nuestro análisis, ya que nos permitirán examinar la relación
entre la resolución de la guía de programación, el éxito académico en la primera evaluación y
el programa de estudio al que pertenecen los estudiantes (véase Tabla \ref{tab:variables}).

\subsubsection{Variable Objetivo}

Dado que el objetivo de esta investigación es determinar si la resolución de la guía tiene algún impacto en la aprobación de la prueba de evaluación del curso, se propone utilizar la variable sol1 como variable objetivo. Dado que esta variable es de tipo cuantitativa, se sugiere generar una nueva columna llamada aprobado, la cual será de tipo binaria. En esta columna, se representará con el valor 1 a las notas superiores a 4.0 hasta la nota máxima de 7.0, mientras que se asignará el valor 0 a aquellas notas inferiores a las indicadas, representando la condición de reprobado.

\subsubsection{Descripción del DataFrame}

La tabla \ref{tab:descripcion_dataframe} presenta una descripción del DataFrame, que incluye información sobre las columnas, el número de valores no nulos y los tipos de datos correspondientes. Analizando esta descripción, podemos obtener una visión general de la estructura y las características del DataFrame.

\begin{table}[H]
    \centering
    \caption{Descripción del DataFrame}
    \begin{tabular}{lll}
        \hline
        \textbf{Columna} & \textbf{Non-Null Count} & \textbf{Dtype} \\
        \hline
        hito1            & 839 non-null            & float64        \\
        hito2            & 839 non-null            & float64        \\
        exitosos         & 839 non-null            & int64          \\
        fallidos         & 839 non-null            & int64          \\
        sol1             & 839 non-null            & float64        \\
        aprobado         & 839 non-null            & int64          \\
        e0 - e52         & 839 non-null            & int64          \\
        \hline
    \end{tabular}%
    \label{tab:descripcion_dataframe}%
\end{table}%

En la tabla \ref{tab:descripcion_dataframe}, cada una de estas columnas tiene un total de 839 valores no nulos y se identifica el tipo de dato correspondiente. Estos detalles son fundamentales para comprender la composición y las propiedades del DataFrame analizado.

\subsubsection{Estadísticas de la variable objetivo}

En el análisis de datos, es importante comprender las características estadísticas de las variables numéricas. En este contexto, hemos examinado la variable sol1 y recopilado estadísticas como el recuento, la media, la desviación estándar, los cuartiles y el sesgo. Estos valores nos proporcionan información sobre la distribución y la tendencia central de la variable.

\begin{table}[H]
    \centering
    \caption{Estadísticas de la variable objetivo}
    \begin{tabular}{ll}
        \hline
        \textbf{Medida}    & \textbf{Valor}       \\
        \hline
        Count              & 839.000000           \\
        Mean               & 3.642789             \\
        Standard Deviation & 1.832625             \\
        Minimum            & 1.000000             \\
        25\% Percentile    & 2.200000             \\
        50\% Percentile    & 3.700000             \\
        75\% Percentile    & 5.100000             \\
        Maximum            & 7.000000             \\
        Skewness           & 0.033079652062595215 \\
        \hline
    \end{tabular}%
    \label{tab:estadistica_variable_sol1}%
\end{table}%

En la tabla \ref{tab:estadistica_variable_sol1}, la variable "sol1" presenta una media de aproximadamente 3.6 y una desviación estándar de alrededor de 1.83. La distribución de los datos muestra una ligera asimetría positiva con un sesgo de aproximadamente 0.03. Estos resultados nos ayudan a comprender la variabilidad y la forma de la distribución de la variable sol1.

\subsubsection{Coeficiente de asimetría variable objetivo}

El coeficiente de asimetría es una medida estadística que nos proporciona información sobre la asimetría de una distribución de datos. En el contexto de los datos analizados, hemos obtenido un coeficiente de asimetría de aproximadamente 3.31\%. Este valor indica una ligera asimetría positiva en la distribución.

\begin{table}[H]
    \centering
    \caption{Coeficiente de asimetría}
    \begin{tabular}{ll}
        \hline
        \textbf{Coeficiente de asimetría}      & \textbf{Valor}       \\
        \hline
        Coeficiente de asimetría               & 0.033079652062595215 \\
        Coeficiente de asimetría en porcentaje & 3.31\%               \\
        \hline
    \end{tabular}%
    \label{tab:skewness}%
\end{table}%

En la tabla \ref{tab:skewness}, se puede apreciar la aproximacion de 0.033079652062595215 lo cual nos indica que la distribución de los datos tiene una ligera asimetría hacia la derecha. Esto implica que hay una cola derecha más larga en comparación con la cola izquierda de la distribución. En términos porcentuales, esta asimetría representa aproximadamente el 3.31\% del rango total de la distribución.

\subsubsection{Coeficiente de Variación variable objetivo}

El coeficiente de variación es una medida de la dispersión relativa de una variable en relación a su media. Nos permite evaluar la variabilidad de los datos en comparación con su valor promedio. Se calcula dividiendo la desviación estándar por la media y se expresa como un porcentaje.

\begin{table}[H]
    \centering
    \caption{Coeficiente de Variación}
    \begin{tabular}{ll}
        \hline
        \textbf{Medida}                        & \textbf{Valor}     \\
        \hline
        Coeficiente de Variación               & 0.5027829289053924 \\
        \hline
        Coeficiente de Variación en Porcentaje & 50.28\%            \\
        \hline
    \end{tabular}%
    \label{tab:coef_variacion}%
\end{table}%

En la tabla \ref{tab:coef_variacion}, se muestra el coeficiente de variación calculado para los datos analizados. El coeficiente de variación es de aproximadamente 0.5027, lo que indica una alta dispersión relativa en relación con la media. Esto se confirma por el coeficiente de variación en porcentaje, que es del 50.28\%. Estos resultados destacan la variabilidad de los datos en el conjunto de datos analizado.

\subsubsection{Obteniendo Amplitud variable variable objetivo}

La amplitud es una medida de la variabilidad o dispersión de los datos. Nos permite evaluar la diferencia entre el valor máximo y mínimo de una variable, proporcionando información sobre la extensión de los datos en el conjunto.

\begin{table}[H]
    \centering
    \caption{Amplitud}
    \begin{tabular}{ll}
        \hline
        \textbf{Medida} & \textbf{Valor}     \\
        \hline
        Amplitud        & 0.5600809456082252 \\
        \hline
    \end{tabular}%
    \label{tab:amplitud}%
\end{table}%

En la tabla \ref{tab:amplitud}, se muestra la amplitud calculada para los datos analizados. La amplitud es de aproximadamente 0.56\%, lo que indica la diferencia entre el valor máximo y mínimo de la variable. Esta medida nos proporciona una idea de la extensión de los datos en el conjunto analizado.

\subsubsection{Tabla de Frecuencias variable objetivo}

Utilizando los datos obtenidos de la tabla \ref{tab:skewness} y la tabla \ref{tab:amplitud}, se ha construido una tabla de frecuencias que muestra la distribución de los datos en intervalos. Los intervalos se definen con base en
la amplitud y el valor máximo de los datos analizados.

\begin{table}[H]
    \centering
    \caption{Tabla de Frecuencias}
    \begin{tabular}{lllll}
        \hline
        \textbf{Intervalo} & \textbf{f\_i} & \textbf{F\_i} & \textbf{h\_i} & \textbf{H\_i} \\
        \hline
        (0.0, 0.56]        & 0             & 0             & 0.000000      & 0.000000      \\
        (0.56, 1.12]       & 152           & 152           & 0.181168      & 0.181168      \\
        (1.12, 1.68]       & 21            & 173           & 0.025030      & 0.206198      \\
        (1.68, 2.24]       & 66            & 239           & 0.078665      & 0.284863      \\
        (2.24, 2.8]        & 79            & 318           & 0.094160      & 0.379023      \\
        (2.8, 3.36]        & 34            & 352           & 0.040524      & 0.419547      \\
        (3.36, 3.92]       & 103           & 455           & 0.122765      & 0.542312      \\
        (3.92, 4.48]       & 76            & 531           & 0.090584      & 0.632896      \\
        (4.48, 5.04]       & 87            & 618           & 0.103695      & 0.736591      \\
        (5.04, 5.6]        & 81            & 699           & 0.096544      & 0.833135      \\
        (5.6, 6.16]        & 53            & 752           & 0.063170      & 0.896305      \\
        (6.16, 6.72]       & 57            & 809           & 0.067938      & 0.964243      \\
        \hline
    \end{tabular}%
    \label{tab:tabla_frecuencias}%
\end{table}%

En la tabla \ref{tab:tabla_frecuencias}, se presenta la tabla de frecuencias que muestra la cantidad de datos en cada intervalo, el total acumulado de datos hasta ese intervalo, la frecuencia relativa del intervalo y la frecuencia
relativa acumulada. Esta tabla nos permite visualizar la distribución de los datos y su acumulación en cada intervalo.

El intervalo más relevante en esta tabla es el intervalo (3.36, 3.92], ya que contiene la mayor frecuencia (103) y la mayor acumulación (455). Esto indica que la mayoría de los datos se encuentran en este rango de valores.

A continuación, se muestra una nueva tabla con información adicional:

\begin{table}[H]
    \centering
    \caption{Información adicional}
    \begin{tabular}{lllll}
        \hline
        \textbf{Mediana} & \textbf{Intervalo de la mediana} & \textbf{Máximo} & \textbf{Intervalo del máximo} \\
        \hline
        3.7              & $f_i$ 103.000000                 & 7.0             & $f_i$ 57.000000               \\
                         & $F_i$ 455.000000                 &                 & $F_i$ 809.000000              \\
                         & $h_i$ 0.122765                   &                 & $h_i$ 0.067938                \\
                         & $H_i$ 0.542312                   &                 & $H_i$ 0.964243                \\
        \hline
    \end{tabular}%
    \label{tab:informacion_adicional}%
\end{table}%

En la tabla \ref{tab:informacion_adicional}, se muestra la mediana de los datos, así como el intervalo en el que se encuentra dicha mediana. Además, se indica el valor máximo y el intervalo en el que se encuentra dicho valor máximo.

\subsubsection{Histograma con curva de densidad variable objetivo}

El histograma con curva de densidad es una herramienta visual importante en el análisis de datos para comprender la distribución de los valores de la variable sol1.

\begin{figure}[H]
    \centering
    \includegraphics[width=4.06111in,height=2.68611in]{img/histogramaConCurvaDeDensidad.png}
    \caption{Histograma con Curva de Densidad}
    \label{fig:hist_density}%
\end{figure}%


La figura \ref{fig:hist_density} muestra este tipo de gráfico correspondiente a los datos analizados.

En el eje Y se presentan los valores de densidad que van desde 0.00 hasta aproximadamente 0.35, representando la densidad de la variable sol1. Por otro lado, en el eje X se muestran los valores del 0 al 7, que representan las diferentes notas obtenidas en la columna sol1 (solemne 1).

La mayor concentración de datos se encuentra alrededor del valor 0.35 en el eje Y, lo cual indica que la mayoría de las observaciones tienen una nota baja en sol1.

La curva de densidad, representada en color rojo, alcanza su punto más alto entre las notas 3 y 4 en el eje X. Esta curva suavizada muestra la forma general de la distribución de los valores de sol1. A medida que las notas aumentan, la densidad disminuye gradualmente.

En resumen, el análisis del histograma con curva de densidad revela que la mayoría de las observaciones tienen una nota baja en sol1, concentrándose principalmente alrededor del valor 0.35 en el eje Y. Además, la curva de densidad muestra un pico más alto entre las notas 3 y 4 en el eje X.

\subsubsection{Identificar valores atípicos variable objetivo}

En el análisis de datos, la detección de valores atípicos es crucial para identificar observaciones que difieren significativamente de la tendencia general. Estos valores pueden tener un impacto significativo y requerir atención especial.

A continuación se muestra una tabla con los valores atípicos identificados mediante el método del Z-score. Se calculó el Z-score para cada observación utilizando un umbral de 3 desviaciones estándar. Los valores que superan este umbral se consideran atípicos y se presentan en la tabla:

\begin{table}[H]
    \centering
    \caption{Valores Atípicos}
    \begin{tabular}{ccccccc}
        \hline
        \textbf{hito1} & \textbf{hito2} & \textbf{exitosos} & \textbf{fallidos} & \textbf{programa} & \textbf{sol1} & \textbf{aprobado} \\
        21.0           & 6.0            & 17                & 14                & UNAB11500         & 1.0           & 0                 \\
        2.0            & 2.0            & 4                 & 27                & UNAB12210         & 1.0           & 0                 \\
        4.0            & 4.0            & 6                 & 41                & UNAB12510         & 1.5           & 0                 \\
        0.0            & 0.0            & 0                 & 47                & UNAB12100         & 1.6           & 0                 \\
        10.0           & 6.0            & 9                 & 38                & UNAB11500         & 1.6           & 0                 \\
        12.0           & 0.0            & 9                 & 38                & UNAB12210         & 2.4           & 0                 \\
        42.0           & 12.0           & 26                & 37                & UNAB21500         & 2.5           & 0                 \\
        32.0           & 32.0           & 26                & 5                 & UNAB11500         & 3.4           & 0                 \\
        9.0            & 0.0            & 7                 & 40                & UNAB11500         & 4.3           & 1                 \\
        38.0           & 6.0            & 28                & 35                & UNAB12210         & 4.4           & 1                 \\
        32.0           & 32.0           & 26                & 5                 & UNAB12210         & 4.6           & 1                 \\
        18.0           & 2.0            & 11                & 20                & UNAB12210         & 4.6           & 1                 \\
        32.0           & 14.0           & 21                & 10                & UNAB12210         & 5.9           & 1                 \\
        13.0           & 25.0           & 16                & 15                & UNAB22115         & 7.0           & 1                 \\
        7.0            & 0.0            & 5                 & 42                & UNAB12100         & 7.0           & 1                 \\
        \hline
    \end{tabular}%
    \label{tab:valores_atipicos}%
\end{table}%

Observando los valores atípicos en la tabla \ref{tab:valores_atipicos}, podemos notar que algunas observaciones difieren significativamente en al menos una de las variables. exitosos representa las respuestas correctas de la guía, fallidos indica la cantidad de errores para lograr los exitosos, programa al que pertenece. sol1 se refiere a la nota obtenida en la solemne, aprobado se refiere si la nota de solemen es mayor mayor o igual a 4 el alumno es aprobado(columna agregada por medio de script). Estos valores atípicos pueden ser de interés para un análisis más detallado, ya que podrían indicar situaciones excepcionales o errores en la recolección de datos.

\subsection{Comparación de algoritmos}

En esta sección, comparamos varios algoritmos con respecto a nuestras métricas de interés: Accuracy, Precision, Recall y F1 Score (R2). Los algoritmos considerados se dividen en:

\subsubsection*{Clasificación:}

\begin{itemize}
    \item DecisionTreeClassifier
    \item LogisticRegression
    \item RandomForestClassifier
    \item XGBClassifier
\end{itemize}

Para estos modelos, se utilizará la variable objetivo \say{aprobado}, la técnica \say{Stratified K-Fold Cross-Validation}, ajustaremos el mejor modelo en los datos de entrenamiento y realizaremos predicciones utilizando el mejor modelo.

La mejor configuración para los modelos de clasificación se muestra en la Figura \ref{fig:config_clasifiacion}:

\begin{figure}[H]
    \centering
    \includegraphics[width=6.06111in,height=2.68611in]{img/compara_algoritmos/configModelsClasificacion.png}
    \caption{Configuración de los modelos de clasificación}
    \label{fig:config_clasifiacion}
\end{figure}

Los resultados obtenidos se presentan en la Figura \ref{fig:metricas_clasificacion}:

\begin{figure}[H]
    \centering
    \includegraphics[width=7.06111in,height=4.68611in]{img/compara_algoritmos/metricasEntreModelosClasificacion.png}
    \caption{Métricas entre modelos de clasificación}
    \label{fig:metricas_clasificacion}
\end{figure}

Se observa que el modelo RandomForestClassifier es el mejor modelo, con un F1 Score del 62.08\%, un recall del 58.67\%, una precisión del 66.54\% y una exactitud del 68.11\%, en comparación con los otros modelos evaluados.

\subsubsection*{Regresión:}

\begin{itemize}
    \item LinearRegression
    \item DecisionTreeRegressor
    \item KNeighborsRegressor
\end{itemize}

Para estos modelos, se utilizará la variable objetivo \say{sol1}, la técnica \say{K-Fold Cross-Validation}, ajustaremos el mejor modelo en los datos de entrenamiento y realizaremos predicciones utilizando el mejor modelo.

La mejor configuración para los modelos de regresión se muestra en la Figura \ref{fig:config_regresion}:

\begin{figure}[H]
    \centering
    \includegraphics[width=5.06111in,height=1.68611in]{img/compara_algoritmos/configModelsRegresion.png}
    \caption{Configuración de los modelos de regresión}
    \label{fig:config_regresion}
\end{figure}

Los resultados obtenidos se presentan en la Figura \ref{fig:metricas_regresion}:

\begin{figure}[H]
    \centering
    \includegraphics[width=7.06111in,height=4.68611in]{img/compara_algoritmos/metricasEntreModelosRegresion.png}
    \caption{Métricas entre modelos de regresión}
    \label{fig:metricas_regresion}
\end{figure}

Se observa que el modelo LinearRegression es el mejor modelo de regresión, con un MSE del 2.77\% y un MAE del 2.77\%, valores inferiores a los de los otros modelos evaluados. Además, el modelo LinearRegression presenta un R2 más cercano a 1, con un aumento del 0.16\% en comparación con los demás modelos.

En conclusión, se realizó una comparación exhaustiva de diferentes algoritmos de modelos predictivos para determinar cuál es el más adecuado en términos de origen de datos. Después de analizar y comparar varios algoritmos, se llegó a la conclusión de que el modelo RandomForestClassifier se destaca como el enfoque más efectivo para el origen de datos en cuestión, mientras que el modelo LinearRegression es el más adecuado para análisis de regresión.

\subsection{Análisis SHAP}

Después de realizar la investigación y la comparación de algoritmos, se determinó que el mejor modelo de clasificación es RandomForestClassifier, y el mejor modelo de regresión es LinearRegression, como el objetivo de la investigacion es determina la aprobacion del ramo el modelo de clasifacion RandomForestClassifier es es el mas adecuardo para realizar nuestro analisis utilizando la variable objetivo \say{aprobado} que nos permitira indentificar si aprobo o reprobo la primera evaluacion y si la guia de apoyo tiene algun impacto significativo para la aprobacion. A continuación, analizaremos los resultados de este modelo.

\subsubsection{Análisis del Mejor Modelo de Clasificación: RandomForestClassifier}


Para el análisis, se prepararon las coordenadas de análisis X/Y utilizando la columna \say{aprobado} (binaria, obtenida de \say{sol1}, donde 1 significa aprobado con una nota mayor o igual a 4) como referencia para el eje Y. Analizaremos el comportamiento de las demás columnas en relación con dicho eje X.

La selección de características y variable objetivo se presenta a continuacion:

\begin{lstlisting}[language=Python, caption=Selección de características y variable objetivo RandomForestClassifier, label=lst:seleccion_caracteristicasRFC]
    y = df["aprobado"]
    X = df[
        ['hito1', 'hito2', 'exitosos', 'fallidos', 'e0', 'e1', 'e2', 'e3', 'e4', 'e5', 'e6', 'e7', 'e8', 'e9', 'e10', 'e11', 'e12', 'e13', 'e14', 'e15', 'e16', 'e17', 'e18', 'e19', 'e20', 'e21', 'e22', 'e23', 'e24', 'e25', 'e26', 'e27', 'e28', 'e29', 'e30', 'e31', 'e32', 'e33', 'e34', 'e35', 'e36', 'e37', 'e38', 'e39', 'e40', 'e41', 'e42', 'e43', 'e44', 'e45', 'e46', 'e47', 'e48', 'e49', 'e50', 'e51', 'e52']
    ]
\end{lstlisting}

Se dividen los datos en un conjunto de entrenamiento (80\% de los datos) y un conjunto de prueba (20\% de los datos), y almacena esos conjuntos en las variables correspondientes. Esto es útil para evaluar el rendimiento del modelo en datos no vistos durante el entrenamiento.

\begin{lstlisting}[language=Python, caption=Division datos de entrenamiento, label=lst:train_test_split_RFC]
    X_train, X_test, y_train, y_test = train_test_split(X,y, test_size = 0.2,random_state= 1502)
\end{lstlisting}

Se realiza del modelo de clasificacion segun la mejor definicion de sus opciones de configuracion con el que se obtuvo el mejor resultado en la comparacion de algoritmos.

\begin{lstlisting}[language=Python, caption=Deficion de modelo RandomForestRegressor, label=lst:def_RFC]
    model = RandomForestRegressor( 
        max_depth=10, 
        min_samples_split=10, 
        min_samples_leaf=5,
        random_state= 1502,
        n_estimators=500
    )
\end{lstlisting}

Realizar Stratified K-Fold Cross-Validation en los datos de entrenamiento

\begin{lstlisting}[language=Python, caption=Realizar Stratified K-Fold Cross-Validation en los datos de entrenamiento, label=lst:skfold_train]
    # Realizar Stratified K-Fold Cross-Validation en los datos de entrenamiento
    from sklearn.model_selection import StratifiedKFold
    
    
    kfold = StratifiedKFold(n_splits=10, shuffle=True, random_state=1502)
    
    # Matrices para almacenar los resultados de validación cruzada
    cv_scores = []
    cv_predictions = []
    
    for train_index, val_index in kfold.split(X_train, y_train):
        # Dividir los datos en conjuntos de entrenamiento y validación
        X_train_fold, X_val_fold = X_train.iloc[train_index], X_train.iloc[val_index]
        y_train_fold, y_val_fold = y_train.iloc[train_index], y_train.iloc[val_index]
        
        # Entrenar el modelo en el conjunto de entrenamiento del fold actual
        model.fit(X_train_fold, y_train_fold)
    
        # Realizar predicciones en el conjunto de validación del fold actual
        y_val_pred = model.predict(X_val_fold)
    
        # Calcular el error cuadrático medio en el conjunto de validación del fold actual
        fold_score = mean_squared_error(y_val_fold, y_val_pred)
        cv_scores.append(fold_score)
    
        # Almacenar las predicciones del fold actual para su uso posterior
        cv_predictions.extend(y_val_pred)
    
        # Calcular el coeficiente de determinación en el conjunto de validación del fold actual
        r2 = r2_score(y_val_fold, y_val_pred)
        print("Fold - Error cuadrático medio:", fold_score)
        print("Fold - Coeficiente de determinación (R2):", r2)
        print()
    
    # Calcular la puntuación promedio de validación cruzada
    avg_score = np.mean(cv_scores)
    percentage_score = avg_score * 100
    print("Promedio del error cuadrático medio en validación cruzada:", avg_score)
    print("Promedio del error cuadrático medio en validación cruzada en %:", percentage_score)
\end{lstlisting}


los resultados obtenidos son los siguientes:

\begin{lstlisting}[language=Python, caption=Realizar Stratified K-Fold Cross-Validation en los datos de entrenamiento, label=lst:skfold_train]
    Fold - Error cuadrático medio: 0.22338734811732283
Fold - Coeficiente de determinación (R2): 0.0994393219751517

Fold - Error cuadrático medio: 0.21610303921931265
Fold - Coeficiente de determinación (R2): 0.13074682521909087

Fold - Error cuadrático medio: 0.23996759796803746
Fold - Coeficiente de determinación (R2): 0.029536443893225295

Fold - Error cuadrático medio: 0.20615353475177942
Fold - Coeficiente de determinación (R2): 0.1662853896389751

Fold - Error cuadrático medio: 0.21340755435714084
Fold - Coeficiente de determinación (R2): 0.13694908873044587

Fold - Error cuadrático medio: 0.212295439508615
Fold - Coeficiente de determinación (R2): 0.14144664148272745

Fold - Error cuadrático medio: 0.2277649508485301
Fold - Coeficiente de determinación (R2): 0.07888570778463855

Fold - Error cuadrático medio: 0.20885878320610823
Fold - Coeficiente de determinación (R2): 0.1553449749439464

Fold - Error cuadrático medio: 0.23418401446153864
...
Fold - Coeficiente de determinación (R2): 0.22574074537220024

Promedio del error cuadrático medio en validación cruzada: 0.21735742055519647
Promedio del error cuadrático medio en validación cruzada en %: 21.735742055519648
\end{lstlisting}

Si el RMSE es 0, significa que el modelo predice perfectamente los valores reales. Cuanto más cerca esté el MSE o RMSE de cero, mejor será el rendimiento del modelo en términos de la diferencia entre las predicciones y los valores reales.

Realizar la predicción con SHAP utilizando el conjunto de prueba

\begin{lstlisting}[language=Python, caption=Realizar Stratified K-Fold Cross-Validation en los datos de entrenamiento, label=lst:skfold_train]
    # Realizar la predicción con SHAP utilizando el conjunto de prueba (X_test)
    explainer = shap.Explainer(model)
    shap_values = explainer.shap_values(X_test)
\end{lstlisting}

A continuacion revisaremos los resultados obtenidos revisando los graficos SHAP.

Para comprender mejor los gráficos que veremos a continuación, es necesario entender lo siguiente:

\begin{itemize}
    \item El término \say{higher} se refiere a las instancias que tienen valores más altos en comparación con otras instancias, y tienen un mayor impacto en la probabilidad de ser \say{aprobado}.
    \item El término \say{lower} se refiere a las instancias con valores más bajos de la característica, que tienen un menor impacto en la probabilidad de ser \say{aprobado}.
\end{itemize}

Viendo los graficos de caracteristicas

\begin{lstlisting}[language=Python, caption=Realizar Stratified K-Fold Cross-Validation en los datos de entrenamiento, label=lst:skfold_train]
    # Mostrar Grafico de Caracteristicas
    shap.summary_plot(shap_values, X_test)
\end{lstlisting}


En la Figura \ref{fig:caract_var_shap} se muestra el gráfico de características variables SHAP gráfico.

\begin{figure}[H]
    \centering
    \includegraphics[width=0.5\textwidth]{img/shap_rf/shapForcePlot2.png}
    \caption{Características Variables SHAP}
    \label{fig:caract_var_shap}
\end{figure}

Al observar este podemos destacar lo siguiente:

\begin{itemize}
    \item La variable \say{hito1}  muestra un alto \say{higher} contribuyendo bastante informacion con sus picos de dispercion.
    \item La pregunta de la guía \say{e29} también es una variable de interés, siendo una de las preguntas de la guía con la mayor dispercion en sesgos \say{higher} y \say{lower}.
    \item Tanto las variables \say{exitosos} como \say{fallidos} también son variables interesantes de analizar, ya que están correlacionadas con la intención de resolver la guía.
    \item La variable \say{e42} como la variable \say{e3}, tiene un un conjunto de datos bastante marcado entre  sus colores \say{azul} y \say{rojo}.
\end{itemize}

Además, se presenta otro gráfico generado por matplotlib sobre la prediccion del modelo en la Figura \ref{fig:caract_var_shap_mat}:

\begin{figure}[H]
    \centering
    \includegraphics[width=1\textwidth]{img/shap_rf/shapForcePlot.png}
    \caption{Características Variables SHAP (Matplotlib)}
    \label{fig:caract_var_shap_mat}
\end{figure}

Al revisar este gráfico, podemos obtener más detalles sobre las características y su importancia:

\begin{itemize}
    \item La sección \say{higher} muestra los valores positivos que contribuyen a aumentar el valor de predicción. En este caso, \say{hito1} esta mas sercana a la \say{f(x)}, lo que significa que esta característica contribuye positivamente al valor de predicción en conjuntos con las variables \say{exitosos}, \say{e42}, \say{e29}, \say{e35}, \say{e3}, \say{fallidos} para \say{aprobado}.
    \item La sección \say{lower} muestra los valores negativos o fallidos que contribuyen a disminuir el valor de predicción. En este caso, la variable \say{e18}, tiene una contribución negativa al valor de prediccióno que no aporta mucho.
    \item En el gráfico, la marca \say{f(x)} representa el valor de predicción del modelo, que en este caso es 0.81.
\end{itemize}

Además de los gráficos anteriores, se presentan los gráficos de dependencia para algunas variables con referencia a \say{aprobado}:

\begin{figure}[H]

    \begin{subfigure}{0.5\textwidth}
        \includegraphics[width=0.9\linewidth, height=6cm]{img/shap_rf/hito1.png}
        \caption{hito1}
        \label{fig:dependencia_hito1}
    \end{subfigure}
    \begin{subfigure}{0.5\textwidth}
        \includegraphics[width=0.9\linewidth, height=6cm]{img/shap_rf/e29.png}
        \caption{e29}
        \label{fig:dependencia_e29}
    \end{subfigure}

    \caption{Variables de dependencias hito1 - e29}
    \label{fig:image2}
\end{figure}

En la Figura \ref{fig:dependencia_hito1} se muestra el gráfico de dependencia para la variable \say{hito1} y refleja \say{fallidos} con su correlación. mientras tanto en la figura \ref{fig:dependencia_e29} se muestra el gráfico de dependencia para la variable \say{e29} con la variable \say{extisoso}.

\begin{figure}[H]

    \begin{subfigure}{0.5\textwidth}
        \includegraphics[width=0.9\linewidth, height=6cm]{img/shap_rf/exitosos.png}
        \caption{exitosos}
        \label{fig:dependencia_exitosos}
    \end{subfigure}
    \begin{subfigure}{0.5\textwidth}
        \includegraphics[width=0.9\linewidth, height=6cm]{img/shap_rf/fallidos.png}
        \caption{fallidos}
        \label{fig:dependencia_fallidos}
    \end{subfigure}

    \caption{Variable de dependencias exitosos - fallidos}
    \label{fig:image2}
\end{figure}

En la Figura \ref{fig:dependencia_exitosos} se muestra el gráfico de dependencia para la variable \say{exitosos} con su variable de dependencia \say{fallidos}, mientras tanto en la figura \ref{fig:dependencia_fallidos} se muestra el gráfico de dependencia para la variable \say{fallidos} con su dependencia \say{exitosos}.

\begin{figure}[H]

    \begin{subfigure}{0.5\textwidth}
        \includegraphics[width=0.9\linewidth, height=6cm]{img/shap_rf/e42.png}
        \caption{e42}
        \label{fig:dependencia_e42}
    \end{subfigure}
    \begin{subfigure}{0.5\textwidth}
        \includegraphics[width=0.9\linewidth, height=6cm]{img/shap_rf/e3.png}
        \caption{e3}
        \label{fig:dependencia_e3}
    \end{subfigure}

    \caption{Variable de dependencias e42 - e3}
    \label{fig:image2}
\end{figure}

En la Figura \ref{fig:dependencia_e42} se muestra el gráfico de dependencia para la variable \say{e42} con su dependencia \say{hito1}, mientras tanto en la figura \ref{fig:dependencia_e3} se muestra el gráfico de dependencia para la variable \say{e3} con la variable \say{e32}.

Estos gráficos de dependencia representan la relación entre los valores de las variables mencionadas y los valores de Shapley en el modelo. Proporcionan una visualización de cómo estas variables influyen en las predicciones del modelo y ayudan a comprender su importancia relativa.

En resumen, podemos entender que el modelo predictivo RandomForestClassifier tiene una función f(x) con una precisión del 81\%. Además, al observar las variables con mayor impacto en la probabilidad de ser clasificado como \textit{aprobado} (denominadas \textit{higher}), encontramos: \textit{hito1}, \textit{exitosos}, \textit{e42}, \textit{e29}, \textit{e35}, \textit{e3} y \textit{fallidos}. Por otro lado, las variables con menor impacto en la probabilidad de ser clasificado como \textit{aprobado} se encuentran en el conjunto \textit{lower}, destacando la variable \textit{e18}.

En resumen, el análisis SHAP nos ha permitido identificar las características y variables que más influyen en los modelos de clasificación. Esto nos brinda una mejor comprensión de los factores que determinan la resolucion de la guia y la nota obtenida en la prueba de conocimientos.






\subsection{Análisis Exploratorio de Causalidad con DoWhy}

\textbf{Objetivo del Análisis Exploratorio de Causalidad:} Tras identificar las características más influyentes en nuestro modelo a través del análisis SHAP, buscamos entender no solo la correlación, sino también las posibles relaciones causales entre estas características y los resultados de los estudiantes. Utilizando la biblioteca DoWhy, que facilita un enfoque basado en gráficos causales, pretendemos desentrañar las verdaderas relaciones causales detrás de las predicciones de nuestro modelo. Esto nos permitirá diseñar intervenciones más efectivas, basadas no solo en correlaciones observadas, sino en relaciones causales validadas.

\textbf{Metodología Utilizada:} En el ámbito del análisis de causalidad, el primer paso es construir un modelo causal, a menudo representado gráficamente, que captura nuestras creencias iniciales sobre las relaciones causales entre las variables. Este modelo se basa tanto en el conocimiento previo como en la lógica, y establece claramente las variables de tratamiento, resultado y las posibles causas comunes (confundidores). Una vez que tenemos este modelo, podemos identificar el efecto causal de interés y estimarlo utilizando datos observacionales.

Sin embargo, solo basarse en datos observacionales para la estimación causal puede ser engañoso debido a posibles sesgos. Aquí es donde entran los refutadores. Los refutadores en DoWhy son técnicas que nos permiten validar la robustez de nuestros hallazgos causales. Al aplicar diferentes refutadores, como la inserción de una causa común no observada o el uso de un tratamiento placebo, podemos evaluar cuán confiables son nuestras estimaciones causales en presencia de posibles sesgos o supuestos no cumplidos. Si nuestras conclusiones causales se mantienen consistentes incluso después de aplicar estos refutadores, ganamos más confianza en la validez de nuestros hallazgos.

\subsubsection{Análisis de la variable \texttt{hito1}}
Contexto y relevancia específica de la variable \texttt{hito1} dentro de este análisis.
El análisis causal es una herramienta poderosa que nos permite desentrañar las relaciones intrínsecas entre las variables en un conjunto de datos. Tras haber interpretado el modelo con SHAP utilizando un \texttt{RandomForestClassifier}, es imperativo adentrarnos en la causalidad de las variables presentes en nuestro conjunto de datos. Emplearemos la biblioteca \texttt{DoWhy} para discernir y cuantificar el efecto causal de las variables \texttt{hito1}, \texttt{exitoso}, \texttt{fallidos}, y \texttt{e29} sobre la variable \texttt{aprobado}. Iniciaremos con \texttt{hito1} y posteriormente abordaremos las demás variables.

\subsubsection{Modelar problema causal}

Para un análisis causal efectivo, es esencial construir un modelo adecuado que represente fielmente las relaciones entre las variables. Enfocaremos nuestros esfuerzos iniciales en la variable \texttt{hito1}.

\begin{figure}[H]
    \centering
    \begin{minipage}{0.48\textwidth}
        \begin{lstlisting}[language=Python, caption=Modelo causal hito1, label=lst:model_causalHito1]
from dowhy import CausalModel

model = CausalModel(
    data=df,
    treatment="hito1",
    outcome="aprobado",
    common_causes=[
        "fallidos",
        "exitosos",
        "e29"
    ],
)
        \end{lstlisting}
    \end{minipage}
    \hfill
    \begin{minipage}{0.48\textwidth}
        \centering
        \includegraphics[width=0.8\textwidth]{img/causalidad/graph_causal_model_hito1.png}
        \caption{Modelo Causal hito1}
        \label{fig:modelo_causal_hito1}
    \end{minipage}
\end{figure}

La visualización del modelo causal a través del método \texttt{view\_model} nos ofrece una representación gráfica de las relaciones causales propuestas entre las variables, facilitando así una comprensión intuitiva de las interacciones entre ellas.

\subsubsection{Identificar y Estimar el efecto causal}

Con el modelo causal en su lugar, el siguiente paso es identificar y cuantificar el efecto causal. Esta etapa nos brinda una perspectiva cuantitativa del impacto de \texttt{hito1} en \texttt{aprobado}.

\begin{minipage}{0.5\textwidth}
    \begin{lstlisting}[language=Python, caption=Identificar y Estimar el efecto causal hito1, label=lst:IdentificarEstimarefectoCausalHito1]
identified_estimand = model.identify_effect(
    proceed_when_unidentifiable=True
)

estimate = model.estimate_effect(
    identified_estimand,
    method_name="backdoor.econml.dml.DML",
    control_value=0,
    treatment_value=1,
    target_units="ate",
    method_params={
        "init_params": {
            "model_y": RandomForestRegressor(),
            "model_t": RandomForestRegressor(),
            "model_final": RandomForestRegressor(
                max_depth=10,
                min_samples_split=10,
                min_samples_leaf=5,
                random_state=1502,
                n_estimators=500,
            ),
            "featurizer": None,
        },
        "fit_params": {},
    },
)
\end{lstlisting}
\end{minipage}
\hfill
\begin{minipage}{0.45\textwidth}
    \begin{table}[H]
        \centering        
        \begin{tabular}{lp{0.6\linewidth}}
            \toprule
            \textbf{Resultado} & \textbf{Valor} \\
            \midrule
            Mean value & 0.14677867241609466 \\
            \bottomrule
        \end{tabular}
        \caption{Resultados del Efecto Causal Hito1}
        \label{tab:efecto_causal_hito1}
    \end{table}
\end{minipage}

El término "Mean Value" denota el valor promedio del efecto estimado de \texttt{hito1} sobre \texttt{aprobado}. En este contexto, un valor de 0.14677867241609466 sugiere que, en promedio, un incremento unitario en \texttt{hito1} se asocia con un aumento del 14.68\% en la probabilidad de que \texttt{aprobado} sea verdadero. Esta interpretación nos brinda una perspectiva valiosa sobre la influencia de \texttt{hito1} en el resultado.

\subsubsection{Refutador de datos aleatorios}

El refutador de datos aleatorios nos permite evaluar cómo se comportaría nuestro estimado si introducimos una causa común aleatoria.

\begin{minipage}{0.5\textwidth}
    \begin{lstlisting}[language=Python, caption=Refutador de datos aleatorios hito1, label=lst:RefutadorDatosAleatoriosHito1]
refute1 = model.refute_estimate(
     identified_estimand, estimate, 
     method_name="random_common_cause"
)
\end{lstlisting}
\end{minipage}
\hfill
\begin{minipage}{0.45\textwidth}
    \begin{table}[H]
        \centering        
        \begin{tabular}{lp{0.6\linewidth}}
            \toprule
            \textbf{Resultado} & \textbf{Valor} \\
            \midrule
            Estimado Original & 0.14677867241609466 \\
            Nuevo Efecto & -0.07422180240608484 \\
            p-value & 0.1200000000000001 \\
            \bottomrule
        \end{tabular}
        \caption{Resultados del Refutador de Datos Aleatorios hito1}
        \label{tab:refutador_datos_aleatorios_hito1}
    \end{table}
\end{minipage}

Este refutador nos proporciona insights sobre la solidez de nuestro estimado en presencia de posibles variables omitidas. La variación en el efecto estimado al introducir una causa común aleatoria, junto con un p-value de 0.12, sugiere que nuestro estimado original es bastante robusto y no es altamente influenciado por variables no observadas.

\subsubsection{Refutador de causa común no observada}

El refutador de causa común no observada evalúa el impacto de una causa común no observada en nuestro estimado.

\begin{minipage}{0.5\textwidth}
    \begin{lstlisting}[language=Python, caption=Refutador de causa común no observada hito1, label=lst:RefutadorCausaComúnNoObservadaHito1]
refute2 = model.refute_estimate(
    identified_estimand,
    estimate,
    method_name="add_unobserved_common_cause",
    confounders_effect_on_treatment="binary_flip",
    confounders_effect_on_outcome="linear",
    effect_strength_on_treatment=0.01,
    effect_strength_on_outcome=0.02,
)
\end{lstlisting}
\end{minipage}
\hfill
\begin{minipage}{0.45\textwidth}
    \begin{table}[H]
        \centering
        \begin{tabular}{lp{0.6\linewidth}}
            \toprule
            \textbf{Resultado} & \textbf{Valor} \\
            \midrule
            Estimado Original & -0.024903693843320803 \\
            Nuevo Efecto & 0.25425759568648637 \\
            \bottomrule
        \end{tabular}
        \caption{Resultados del Refutador de Causa Común No Observada hito1}
        \label{tab:refutador_causa_no_observada_hito1}
    \end{table}
\end{minipage}

La introducción de una causa común no observada cambia drásticamente nuestro estimado, pasando de un valor negativo a uno positivo. Esto sugiere que nuestro estimado podría ser sensible a variables no observadas. Es crucial tener en cuenta este tipo de sensibilidades al interpretar los resultados, ya que las variables no observadas pueden introducir sesgos en el análisis causal.

\subsubsection{Refutador de tratamiento placebo}

El refutador de tratamiento placebo evalúa el efecto de un tratamiento ficticio en nuestro estimado.

\begin{minipage}{0.5\textwidth}
    \begin{lstlisting}[language=Python, caption=Refutador de tratamiento placebo hito1, label=lst:RefutadorTratamientoPlaceboHito1]
refute3 = model.refute_estimate(
    identified_estimand,
    estimate,
    method_name="placebo_treatment_refuter",
    placebo_type="permute",
)
\end{lstlisting}
\end{minipage}
\hfill
\begin{minipage}{0.45\textwidth}
    \begin{table}[H]
        \centering
        \begin{tabular}{lp{0.6\linewidth}}
            \toprule
            \textbf{Resultado} & \textbf{Valor} \\
            \midrule
            Estimado Original & -0.024903693843320803 \\
            Nuevo Efecto & 0.000123456789012345 \\
            p-value & 0.99 \\
            \bottomrule
        \end{tabular}
        \caption{Resultados del Refutador de Tratamiento Placebo hito1}
        \label{tab:refutador_placebo_hito1}
    \end{table}
\end{minipage}

Este refutador nos ayuda a discernir si el tratamiento real, \texttt{hito1}, tiene un impacto genuino sobre el resultado o si los efectos observados son meramente aleatorios.

El nuevo efecto, cercano a cero, junto con un p-value de 0.99, sugiere que el tratamiento real (\texttt{hito1}) no tiene un efecto significativo sobre el resultado. Esto indica que los resultados obtenidos inicialmente podrían ser atribuidos al azar y no necesariamente a la influencia de \texttt{hito1} sobre \texttt{aprobado}.


% Después de analizar las variables \texttt{hito1}, \texttt{exitosos} y \texttt{fallidos}, nos enfocamos en la variable \texttt{e29}. A continuación, presentamos la construcción del modelo causal para \texttt{e29} y los resultados obtenidos.

% \begin{figure}[H]
%     \centering
%     \begin{minipage}{0.48\textwidth}
%         \begin{lstlisting}[language=Python, caption=Modelo causal e29, label=lst:model_causalE29]
% from dowhy import CausalModel

% model = CausalModel(
%     data=df,
%     treatment="e29",
%     outcome="aprobado",
%     common_causes=[
%         "fallidos",
%         "exitosos",
%         "hito1"
%     ],
% )
%         \end{lstlisting}
%     \end{minipage}
%     \hfill
%     \begin{minipage}{0.48\textwidth}
%         \centering
%         \includegraphics[width=0.8\textwidth]{img/causalidad/graph_causal_model_e29.png}
%         \caption{Modelo Causal e29}
%         \label{fig:modelo_causal_e29}
%     \end{minipage}
% \end{figure}

% \textbf{Identificar y Estimar el efecto causal}

% Utilizando el mismo código que en \texttt{hito1} (ver \ref{lst:IdentificarEstimarefectoCausalHito1}), obtuvimos los siguientes resultados:

% \begin{table}[H]
%     \centering        
%     \begin{tabular}{lp{0.6\linewidth}}
%         \toprule
%         \textbf{Resultado} & \textbf{Valor} \\
%         \midrule
%         Mean value & 0.1108420573840858 \\
%         \bottomrule
%     \end{tabular}
%     \caption{Resultados del Efecto Causal e29}
%     \label{tab:efecto_causal_e29}
% \end{table}

% El término "Mean Value" denota el valor promedio del efecto estimado de \texttt{e29} sobre \texttt{aprobado}. Un valor de 0.1108420573840858 sugiere que, en promedio, un incremento unitario en \texttt{e29} se asocia con un aumento del 11.08\% en la probabilidad de que \texttt{aprobado} sea verdadero.

% \textbf{Refutador de datos aleatorios}

% Utilizando el mismo código que en \texttt{hito1} (ver \ref{lst:RefutadorDatosAleatoriosHito1}), los resultados fueron:

% \begin{table}[H]
%     \centering        
%     \begin{tabular}{lp{0.6\linewidth}}
%         \toprule
%         \textbf{Resultado} & \textbf{Valor} \\
%         \midrule
%         Estimado Original & 0.1108420573840858 \\
%         Nuevo Efecto & 0.1407657769498767 \\
%         p-value & 0.88 \\
%         \bottomrule
%     \end{tabular}
%     \caption{Resultados del Refutador de Datos Aleatorios e29}
%     \label{tab:refutador_datos_aleatorios_e29}
% \end{table}

% La variación en el efecto estimado al introducir una causa común aleatoria, junto con un p-value de 0.88, sugiere que nuestro estimado original es bastante robusto y no es altamente influenciado por variables no observadas.

% \textbf{Refutador de causa común no observada}

% Utilizando el mismo código que en \texttt{hito1} (ver \ref{lst:RefutadorCausaComúnNoObservadaHito1}), los resultados fueron:

% \begin{table}[H]
%     \centering
%     \begin{tabular}{lp{0.6\linewidth}}
%         \toprule
%         \textbf{Resultado} & \textbf{Valor} \\
%         \midrule
%         Estimado Original & 0.1108420573840858 \\
%         Nuevo Efecto & 0.08324099555164001 \\
%         \bottomrule
%     \end{tabular}
%     \caption{Resultados del Refutador de Causa Común No Observada e29}
%     \label{tab:refutador_causa_no_observada_e29}
% \end{table}

% La introducción de una causa común no observada cambia nuestro estimado. Esto sugiere que nuestro estimado podría ser sensible a variables no observadas.

% \textbf{Refutador de tratamiento placebo}

% Utilizando el mismo código que en \texttt{hito1} (ver \ref{lst:RefutadorTratamientoPlaceboHito1}), los resultados fueron:

% \begin{table}[H]
%     \centering
%     \begin{tabular}{lp{0.6\linewidth}}
%         \toprule
%         \textbf{Resultado} & \textbf{Valor} \\
%         \midrule
%         Estimado Original & 0.1108420573840858 \\
%         Nuevo Efecto & -0.025440763654296695 \\
%         p-value & 0.8400000000000001 \\
%         \bottomrule
%     \end{tabular}
%     \caption{Resultados del Refutador de Tratamiento Placebo e29}
%     \label{tab:refutador_placebo_29}
% \end{table}

% El nuevo efecto, cercano a cero, junto con un p-value de 0.84, sugiere que el tratamiento real (\texttt{e29}) no tiene un efecto significativo sobre el resultado. Esto indica que los resultados obtenidos inicialmente podrían ser atribuidos al azar y no necesariamente a la influencia de \texttt{e29} sobre \texttt{aprobado}.

% \textbf{Conclusión para \texttt{e29}}

% El análisis exploratorio de causalidad con \texttt{DoWhy} para la variable \texttt{e29} nos ha proporcionado insights valiosos sobre su efecto causal en \texttt{aprobado}. A medida que avanzamos en nuestra investigación, estos análisis nos ayudarán a tomar decisiones informadas y a entender mejor las relaciones causales entre las variables.

\subsubsection{Análisis exploratorio de la variable \texttt{exitosos}}
Contexto y relevancia específica de la variable \texttt{exitosos} dentro de este análisis.


Después de analizar la variable \texttt{hito1}, nos enfocamos en la variable \texttt{exitosos}. A continuación, presentamos la construcción del modelo causal para \texttt{exitosos} y los resultados obtenidos.

\begin{figure}[H]
    \centering
    \begin{minipage}{0.48\textwidth}
        \begin{lstlisting}[language=Python, caption=Modelo causal exitosos, label=lst:model_causalExitosos]
from dowhy import CausalModel

model = CausalModel(
    data=df,
    treatment="exitosos",
    outcome="aprobado",
    common_causes=[
        "fallidos",
        "hito1",
        "e29"
    ],
)
        \end{lstlisting}
    \end{minipage}
    \hfill
    \begin{minipage}{0.48\textwidth}
        \centering
        \includegraphics[width=0.8\textwidth]{img/causalidad/graph_causal_model_exitosos.png}
        \caption{Modelo Causal Exitosos}
        \label{fig:modelo_causal_exitosos}
    \end{minipage}
\end{figure}

\textbf{Identificar y Estimar el efecto causal}

Utilizando el mismo código que en \texttt{hito1} (ver \ref{lst:IdentificarEstimarefectoCausalHito1}), obtuvimos los siguientes resultados:

\begin{table}[H]
    \centering        
    \begin{tabular}{lp{0.6\linewidth}}
        \toprule
        \textbf{Resultado} & \textbf{Valor} \\
        \midrule
        Mean value & -0.24676503920081228 \\
        \bottomrule
    \end{tabular}
    \caption{Resultados del Efecto Causal exitosos}
    \label{tab:efecto_causal_exitosos}
\end{table}

El término "Mean Value" denota el valor promedio del efecto estimado de \texttt{exitosos} sobre \texttt{aprobado}. Un valor de -0.24676503920081228 sugiere que, en promedio, un incremento unitario en \texttt{exitosos} se asocia con una disminución del 24.68\% en la probabilidad de que \texttt{aprobado} sea verdadero.

\textbf{Refutador de datos aleatorios}

Utilizando el mismo código que en \texttt{hito1} (ver \ref{lst:RefutadorDatosAleatoriosHito1}), los resultados fueron:

\begin{table}[H]
    \centering        
    \begin{tabular}{lp{0.6\linewidth}}
        \toprule
        \textbf{Resultado} & \textbf{Valor} \\
        \midrule
        Estimado Original & -0.24676503920081228 \\
        Nuevo Efecto & -0.08017355855007074 \\
        p-value & 0.38 \\
        \bottomrule
    \end{tabular}
    \caption{Resultados del Refutador de Datos Aleatorios exitosos}
    \label{tab:refutador_datos_aleatorios_exitosos}
\end{table}

La variación en el efecto estimado al introducir una causa común aleatoria, junto con un p-value de 0.38, sugiere que nuestro estimado original es bastante robusto y no es altamente influenciado por variables no observadas.

\textbf{Refutador de causa común no observada}

Utilizando el mismo código que en \texttt{hito1} (ver \ref{lst:RefutadorCausaComúnNoObservadaHito1}), los resultados fueron:

\begin{table}[H]
    \centering
    \begin{tabular}{lp{0.6\linewidth}}
        \toprule
        \textbf{Resultado} & \textbf{Valor} \\
        \midrule
        Estimado Original & -0.24676503920081228 \\
        Nuevo Efecto & -0.26835476106611056 \\
        \bottomrule
    \end{tabular}
    \caption{Resultados del Refutador de Causa Común No Observada exitosos}
    \label{tab:refutador_causa_no_observada_exitosos}
\end{table}

La introducción de una causa común no observada cambia ligeramente nuestro estimado. Esto sugiere que nuestro estimado podría ser sensible a variables no observadas.

\textbf{Refutador de tratamiento placebo}

Utilizando el mismo código que en \texttt{hito1} (ver \ref{lst:RefutadorTratamientoPlaceboHito1}), los resultados fueron:

\begin{table}[H]
    \centering
    \begin{tabular}{lp{0.6\linewidth}}
        \toprule
        \textbf{Resultado} & \textbf{Valor} \\
        \midrule
        Estimado Original & -0.24676503920081228 \\
        Nuevo Efecto & 0.01707981451338608 \\
        p-value & 0.96 \\
        \bottomrule
    \end{tabular}
    \caption{Resultados del Refutador de Tratamiento Placebo exitosos}
    \label{tab:refutador_placebo_exitosos}
\end{table}

El nuevo efecto, cercano a cero, junto con un p-value de 0.96, sugiere que el tratamiento real (\texttt{exitosos}) no tiene un efecto significativo sobre el resultado. Esto indica que los resultados obtenidos inicialmente podrían ser atribuidos al azar y no necesariamente a la influencia de \texttt{exitosos} sobre \texttt{aprobado}.

\textbf{Conclusión para \texttt{exitosos}}

El análisis exploratorio de causalidad con \texttt{DoWhy} para la variable \texttt{exitosos} nos ha proporcionado insights valiosos sobre su efecto causal en \texttt{aprobado}. A medida que avanzamos en nuestra investigación, continuaremos analizando las variables \texttt{fallidos} y \texttt{e29} siguiendo un enfoque similar.

\subsubsection{Análisis exploratorio de la variable \texttt{fallidos}}

Después de analizar las variables \texttt{hito1} y \texttt{exitosos}, nos enfocamos en la variable \texttt{fallidos}. A continuación, presentamos la construcción del modelo causal para \texttt{fallidos} y los resultados obtenidos.

\begin{figure}[H]
    \centering
    \begin{minipage}{0.48\textwidth}
        \begin{lstlisting}[language=Python, caption=Modelo causal fallidos, label=lst:model_causalFallidos]
from dowhy import CausalModel

model = CausalModel(
    data=df,
    treatment="fallidos",
    outcome="aprobado",
    common_causes=[
        "hito1",
        "exitosos",
        "e29"
    ],
)
        \end{lstlisting}
    \end{minipage}
    \hfill
    \begin{minipage}{0.48\textwidth}
        \centering
        \includegraphics[width=0.8\textwidth]{img/causalidad/graph_causal_model_fallidos.png}
        \caption{Modelo Causal Fallidos}
        \label{fig:modelo_causal_Fallidos}
    \end{minipage}
\end{figure}

\textbf{Identificar y Estimar el efecto causal}

Utilizando el mismo código que en \texttt{hito1} (ver \ref{lst:IdentificarEstimarefectoCausalHito1}), obtuvimos los siguientes resultados:

\begin{table}[H]
    \centering        
    \begin{tabular}{lp{0.6\linewidth}}
        \toprule
        \textbf{Resultado} & \textbf{Valor} \\
        \midrule
        Mean value & 0.031651437521335604 \\
        \bottomrule
    \end{tabular}
    \caption{Resultados del Efecto Causal Fallidos}
    \label{tab:efecto_causal_Fallidos}
\end{table}

El término "Mean Value" denota el valor promedio del efecto estimado de \texttt{fallidos} sobre \texttt{aprobado}. Un valor de 0.031651437521335604 sugiere que, en promedio, un incremento unitario en \texttt{fallidos} se asocia con un aumento del 3.17\% en la probabilidad de que \texttt{aprobado} sea verdadero.

\textbf{Refutador de datos aleatorios}

Utilizando el mismo código que en \texttt{hito1} (ver \ref{lst:RefutadorDatosAleatoriosHito1}), los resultados fueron:

\begin{table}[H]
    \centering        
    \begin{tabular}{lp{0.6\linewidth}}
        \toprule
        \textbf{Resultado} & \textbf{Valor} \\
        \midrule
        Estimado Original & 0.031651437521335604 \\
        Nuevo Efecto & 0.013599125681968194 \\
        p-value & 0.94 \\
        \bottomrule
    \end{tabular}
    \caption{Resultados del Refutador de Datos Aleatorios Fallidos}
    \label{tab:refutador_datos_aleatorios_Fallidos}
\end{table}

La variación en el efecto estimado al introducir una causa común aleatoria, junto con un p-value de 0.94, sugiere que nuestro estimado original es bastante robusto y no es altamente influenciado por variables no observadas.

\textbf{Refutador de causa común no observada}

Utilizando el mismo código que en \texttt{hito1} (ver \ref{lst:RefutadorCausaComúnNoObservadaHito1}), los resultados fueron:

\begin{table}[H]
    \centering
    \begin{tabular}{lp{0.6\linewidth}}
        \toprule
        \textbf{Resultado} & \textbf{Valor} \\
        \midrule
        Estimado Original & 0.031651437521335604 \\
        Nuevo Efecto & 0.16595441857221216 \\
        \bottomrule
    \end{tabular}
    \caption{Resultados del Refutador de Causa Común No Observada Fallidos}
    \label{tab:refutador_causa_no_observada_fallidos}
\end{table}

La introducción de una causa común no observada cambia significativamente nuestro estimado. Esto sugiere que nuestro estimado podría ser sensible a variables no observadas.

\textbf{Refutador de tratamiento placebo}

Utilizando el mismo código que en \texttt{hito1} (ver \ref{lst:RefutadorTratamientoPlaceboHito1}), los resultados fueron:

\begin{table}[H]
    \centering
    \begin{tabular}{lp{0.6\linewidth}}
        \toprule
        \textbf{Resultado} & \textbf{Valor} \\
        \midrule
        Estimado Original & 0.031651437521335604 \\
        Nuevo Efecto & 0.0019291403589726274 \\
        p-value & 0.98 \\
        \bottomrule
    \end{tabular}
    \caption{Resultados del Refutador de Tratamiento Placebo Fallidos}
    \label{tab:refutador_placebo_Fallidos}
\end{table}

El nuevo efecto, cercano a cero, junto con un p-value de 0.98, sugiere que el tratamiento real (\texttt{fallidos}) no tiene un efecto significativo sobre el resultado. Esto indica que los resultados obtenidos inicialmente podrían ser atribuidos al azar y no necesariamente a la influencia de \texttt{fallidos} sobre \texttt{aprobado}.

\textbf{Conclusión para \texttt{fallidos}}

El análisis exploratorio de causalidad con \texttt{DoWhy} para la variable \texttt{fallidos} nos ha proporcionado insights valiosos sobre su efecto causal en \texttt{aprobado}. A medida que avanzamos en nuestra investigación, continuaremos analizando la variable \texttt{e29} siguiendo un enfoque similar.

\subsubsection{Análisis exploratorio de la variable \texttt{e29}}

Después de analizar las variables \texttt{hito1}, \texttt{exitosos} y \texttt{fallidos}, nos enfocamos en la variable \texttt{e29}. A continuación, presentamos la construcción del modelo causal para \texttt{e29} y los resultados obtenidos.

\begin{figure}[H]
    \centering
    \begin{minipage}{0.48\textwidth}
        \begin{lstlisting}[language=Python, caption=Modelo causal e29, label=lst:model_causalE29]
from dowhy import CausalModel

model = CausalModel(
    data=df,
    treatment="e29",
    outcome="aprobado",
    common_causes=[
        "fallidos",
        "exitosos",
        "hito1"
    ],
)
        \end{lstlisting}
    \end{minipage}
    \hfill
    \begin{minipage}{0.48\textwidth}
        \centering
        \includegraphics[width=0.8\textwidth]{img/causalidad/graph_causal_model_e29.png}
        \caption{Modelo Causal e29}
        \label{fig:modelo_causal_e29}
    \end{minipage}
\end{figure}

\textbf{Identificar y Estimar el efecto causal}

Utilizando el mismo código que en \texttt{hito1} (ver \ref{lst:IdentificarEstimarefectoCausalHito1}), obtuvimos los siguientes resultados:

\begin{table}[H]
    \centering        
    \begin{tabular}{lp{0.6\linewidth}}
        \toprule
        \textbf{Resultado} & \textbf{Valor} \\
        \midrule
        Mean value & 0.1108420573840858 \\
        \bottomrule
    \end{tabular}
    \caption{Resultados del Efecto Causal e29}
    \label{tab:efecto_causal_e29}
\end{table}

El término "Mean Value" denota el valor promedio del efecto estimado de \texttt{e29} sobre \texttt{aprobado}. Un valor de 0.1108420573840858 sugiere que, en promedio, un incremento unitario en \texttt{e29} se asocia con un aumento del 11.08\% en la probabilidad de que \texttt{aprobado} sea verdadero.

\textbf{Refutador de datos aleatorios}

Utilizando el mismo código que en \texttt{hito1} (ver \ref{lst:RefutadorDatosAleatoriosHito1}), los resultados fueron:

\begin{table}[H]
    \centering        
    \begin{tabular}{lp{0.6\linewidth}}
        \toprule
        \textbf{Resultado} & \textbf{Valor} \\
        \midrule
        Estimado Original & 0.1108420573840858 \\
        Nuevo Efecto & 0.1407657769498767 \\
        p-value & 0.88 \\
        \bottomrule
    \end{tabular}
    \caption{Resultados del Refutador de Datos Aleatorios e29}
    \label{tab:refutador_datos_aleatorios_e29}
\end{table}

La variación en el efecto estimado al introducir una causa común aleatoria, junto con un p-value de 0.88, sugiere que nuestro estimado original es bastante robusto y no es altamente influenciado por variables no observadas.

\textbf{Refutador de causa común no observada}

Utilizando el mismo código que en \texttt{hito1} (ver \ref{lst:RefutadorCausaComúnNoObservadaHito1}), los resultados fueron:

\begin{table}[H]
    \centering
    \begin{tabular}{lp{0.6\linewidth}}
        \toprule
        \textbf{Resultado} & \textbf{Valor} \\
        \midrule
        Estimado Original & 0.1108420573840858 \\
        Nuevo Efecto & 0.08324099555164001 \\
        \bottomrule
    \end{tabular}
    \caption{Resultados del Refutador de Causa Común No Observada e29}
    \label{tab:refutador_causa_no_observada_e29}
\end{table}

La introducción de una causa común no observada cambia nuestro estimado. Esto sugiere que nuestro estimado podría ser sensible a variables no observadas.

\textbf{Refutador de tratamiento placebo}

Utilizando el mismo código que en \texttt{hito1} (ver \ref{lst:RefutadorTratamientoPlaceboHito1}), los resultados fueron:

\begin{table}[H]
    \centering
    \begin{tabular}{lp{0.6\linewidth}}
        \toprule
        \textbf{Resultado} & \textbf{Valor} \\
        \midrule
        Estimado Original & 0.1108420573840858 \\
        Nuevo Efecto & -0.025440763654296695 \\
        p-value & 0.8400000000000001 \\
        \bottomrule
    \end{tabular}
    \caption{Resultados del Refutador de Tratamiento Placebo e29}
    \label{tab:refutador_placebo_29}
\end{table}

El nuevo efecto, cercano a cero, junto con un p-value de 0.84, sugiere que el tratamiento real (\texttt{e29}) no tiene un efecto significativo sobre el resultado. Esto indica que los resultados obtenidos inicialmente podrían ser atribuidos al azar y no necesariamente a la influencia de \texttt{e29} sobre \texttt{aprobado}.

\textbf{Conclusión para \texttt{e29}}

El análisis exploratorio de causalidad con \texttt{DoWhy} para la variable \texttt{e29} nos ha proporcionado insights valiosos sobre su efecto causal en \texttt{aprobado}. A medida que avanzamos en nuestra investigación, estos análisis nos ayudarán a tomar decisiones informadas y a entender mejor las relaciones causales entre las variables.

\subsubsection{Reflexión Final del Análisis Exploratorio} Conclusiones derivadas de este análisis y recomendaciones o pasos a seguir.


\textbf{Conclusión de los Análisis por Variable:} A través del análisis de causalidad utilizando \texttt{DoWhy}, hemos profundizado en las relaciones causales de las variables identificadas como significativas en el análisis SHAP. Es evidente que variables como \texttt{hito1}, \texttt{exitosos}, \texttt{fallidos} y \texttt{e29} no solo tienen importancia predictiva, sino que también tienen relaciones causales significativas que influyen en los resultados de los estudiantes. Estos insights causales proporcionan una base más sólida para las intervenciones educativas, permitiendo acciones más dirigidas y efectivas.

\textbf{Reflexión Final del Análisis Exploratorio:} Este análisis exploratorio de causalidad ha complementado y enriquecido nuestra comprensión obtenida del análisis SHAP. Al identificar relaciones causales, no solo correlaciones, estamos mejor posicionados para diseñar intervenciones y estrategias educativas que tengan un impacto real y positivo en el rendimiento de los estudiantes. Sin embargo, es crucial recordar que estos hallazgos, aunque prometedores, están basados en supuestos. Por lo tanto, es esencial validar estos resultados en diferentes contextos o con datos adicionales. La causalidad es compleja, y si bien las herramientas como \texttt{DoWhy} ofrecen un camino para desentrañarla, siempre es prudente abordar los resultados con un grado de cautela y con la mente abierta a futuras investigaciones y validaciones.


