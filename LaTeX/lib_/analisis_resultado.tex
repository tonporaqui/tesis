\hypertarget{analisis_resultado}{%
    \section{Análisis Resultado}\label{Análisis Resultado}}

En nuestro analisis de resultado, se revisará la relación entre la resolución
de la guía de programación y el éxito académico en el ramo de introducción a la
programación. Se analizará específicamente la columna exitosos del conjunto de
datos dataset a 2021, con el objetivo de responder a la pregunta de
investigación planteada en este estudio: ¿La resolución de una guía de
programación está relacionada con el éxito académico en el ramo de introducción
a la programación en la Universidad Andrés Bello? De esta manera, se podrá
determinar si resolver la guía es actualmente significativo para predecir el
rendimiento en la solemne 1 del año 2021,

\subsection{Descripción del DataFrame}

La tabla \ref{tab:descripcion_dataframe} presenta una descripción del DataFrame, que incluye información sobre
las columnas, el número de valores no nulos y los tipos de datos
correspondientes. Analizando esta descripción, podemos obtener una visión
general de la estructura y las características del DataFrame.

\begin{table}[htbp]
    \centering
    \caption{Descripción del DataFrame}
    \begin{tabular}{lll}
        \hline
        \textbf{Columna} & \textbf{Non-Null Count} & \textbf{Dtype} \\
        \hline
        hito1     & 839 non-null    & float64 \\
        hito2     & 839 non-null    & float64 \\
        exitosos  & 839 non-null    & int64   \\
        fallidos  & 839 non-null    & int64   \\
        sol1      & 839 non-null    & float64 \\
        aprobado  & 839 non-null    & int64   \\
        e0 - e52   & 839 non-null    & int64   \\
        \hline
    \end{tabular}
    \label{tab:descripcion_dataframe}
\end{table}

En la tabla \ref{tab:descripcion_dataframe}, cada una de estas columnas tiene un total de
839 valores no nulos y se identifica el tipo de dato correspondiente. Estos
detalles son fundamentales para comprender la composición y las propiedades del
DataFrame analizado.

\subsection{Estadísticas de la variable sol1}

En el análisis de datos, es importante comprender las características
estadísticas de las variables numéricas. En este contexto, hemos examinado la
variable sol1 y recopilado estadísticas como el recuento, la media, la
desviación estándar, los cuartiles y el sesgo. Estos valores nos proporcionan
información sobre la distribución y la tendencia central de la variable.

En la tabla \ref{tab:estadistica_variable_sol1}, la variable "sol1" presenta
una media de aproximadamente 7.48 y una desviación estándar de alrededor de
5.36. La distribución de los datos muestra una ligera asimetría positiva con un
sesgo de aproximadamente 0.19. Estos resultados nos ayudan a comprender la
variabilidad y la forma de la distribución de la variable sol1.

\begin{table}[htbp]
    \centering
    \caption{Estadísticas de la variable sol1}
    \begin{tabular}{ll}
        \hline
        \textbf{Medida}    & \textbf{Valor}      \\
        \hline
        Count              & 839.000000          \\
        Mean               & 7.476758            \\
        Standard Deviation & 5.361101            \\
        Minimum            & 0.000000            \\
        25\% Percentile    & 3.000000            \\
        50\% Percentile    & 7.000000            \\
        75\% Percentile    & 11.000000           \\
        Maximum            & 28.000000           \\
        Skewness           & 0.18644798387069741 \\
        \hline
    \end{tabular}%
    \label{tab:estadistica_variable_sol1}%
\end{table}%



\subsection{Coeficiente de asimetría}

El coeficiente de asimetría es una medida estadística que nos proporciona
información sobre la asimetría de una distribución de datos. En el contexto de
los datos analizados, hemos obtenido un coeficiente de asimetría de
aproximadamente 0.1864. Este valor indica una ligera asimetría positiva en la
distribución.

\begin{table}[htbp]
    \centering
    \caption{Coeficiente de asimetría}
    \begin{tabular}{ll}
        \hline
        \textbf{Coeficiente de asimetría}      & \textbf{Valor}      \\
        \hline
        Coeficiente de asimetría               & 0.18644798387069741 \\
        Coeficiente de asimetría en porcentaje & 18.64\%             \\
        \hline
    \end{tabular}%
    \label{tab:skewness}%
\end{table}%

En la tabla \ref{tab:skewness}, se puede apreciar la aproximacion de 0.1864 lo
cual nos indica que la distribución de los datos tiene una ligera asimetría
hacia la derecha. Esto implica que hay una cola derecha más larga en
comparación con la cola izquierda de la distribución. En términos porcentuales,
esta asimetría representa aproximadamente el 18.64\% del rango total de la
distribución.

\subsection{Coeficiente de Variación}

El coeficiente de variación es una medida de la dispersión relativa de una
variable en relación a su media. Nos permite evaluar la variabilidad de los
datos en comparación con su valor promedio. Se calcula dividiendo la desviación
estándar por la media y se expresa como un porcentaje.

\begin{table}[htbp]
    \centering
    \caption{Coeficiente de Variación}
    \begin{tabular}{ll}
        \hline
        \textbf{Medida}                        & \textbf{Valor}     \\
        \hline
        Coeficiente de Variación               & 0.7166080688736847 \\
        \hline
        Coeficiente de Variación en Porcentaje & 71.66\%            \\
        \hline
    \end{tabular}
    \label{tab:coef_variacion}
\end{table}

En la tabla \ref{tab:coef_variacion}, se muestra el coeficiente de variación
calculado para los datos analizados. El coeficiente de variación es de
aproximadamente 0.7166, lo que indica una alta dispersión relativa en relación
con la media. Esto se confirma por el coeficiente de variación en porcentaje,
que es del 71.66\%. Estos resultados destacan la variabilidad de los datos en
el conjunto de datos analizado.

\subsection{Obteniendo Amplitud}

La amplitud es una medida de la variabilidad o dispersión de los datos. Nos
permite evaluar la diferencia entre el valor máximo y mínimo de una variable,
proporcionando información sobre la extensión de los datos en el conjunto.

\begin{table}[htbp]
    \centering
    \caption{Amplitud}
    \begin{tabular}{ll}
        \hline
        \textbf{Medida} & \textbf{Valor}     \\
        \hline
        Amplitud        & 2.6137110795050504 \\
        \hline
    \end{tabular}
    \label{tab:amplitud}
\end{table}

En la tabla \ref{tab:amplitud}, se muestra la amplitud calculada para los datos
analizados. La amplitud es de aproximadamente 2.6137, lo que indica la
diferencia entre el valor máximo y mínimo de la variable. Esta medida nos
proporciona una idea de la extensión de los datos en el conjunto analizado.

\subsection{Tabla de Frecuencias}

Utilizando los datos obtenidos de la tabla \ref{tab:skewness} y la tabla
\ref{tab:amplitud}, se ha construido una tabla de frecuencias que muestra la
distribución de los datos en intervalos. Los intervalos se definen con base en
la amplitud y el valor máximo de los datos analizados.

\begin{table}[htbp]
    \centering
    \caption{Tabla de Frecuencias}
    \begin{tabular}{lllll}
        \hline
        \textbf{Intervalo} & \textbf{f\_i} & \textbf{F\_i} & \textbf{h\_i} & \textbf{H\_i} \\
        \hline
        (0.0, 2.61]        & 24            & 24            & 0.028605      & 0.028605      \\
        (2.61, 5.22]       & 96            & 120           & 0.114422      & 0.143027      \\
        (5.22, 7.83]       & 128           & 248           & 0.152563      & 0.295590      \\
        (7.83, 10.44]      & 161           & 409           & 0.191895      & 0.487485      \\
        (10.44, 13.05]     & 135           & 544           & 0.160906      & 0.648391      \\
        (13.05, 15.66]     & 48            & 592           & 0.057211      & 0.705602      \\
        (15.66, 18.27]     & 69            & 661           & 0.082241      & 0.787843      \\
        (18.27, 20.88]     & 1             & 662           & 0.001192      & 0.789035      \\
        (20.88, 23.49]     & 1             & 663           & 0.001192      & 0.790226      \\
        (23.49, 26.1]      & 3             & 666           & 0.003576      & 0.793802      \\
        \hline
    \end{tabular}
    \label{tab:tabla_frecuencias}
\end{table}

En la tabla \ref{tab:tabla_frecuencias}, se presenta la tabla de frecuencias
que muestra la cantidad de datos en cada intervalo, el total acumulado de datos
hasta ese intervalo, la frecuencia relativa del intervalo y la frecuencia
relativa acumulada. Esta tabla nos permite visualizar la distribución de los
datos y su acumulación en cada intervalo.

El intervalo más relevante en esta tabla es el intervalo (7.83, 10.44], ya que
contiene la mayor frecuencia (161) y la mayor acumulación (409). Esto indica
que la mayoría de los datos se encuentran en este rango de valores.

\subsection{Histograma con curva de densidad}

El histograma con curva de densidad es una herramienta visual importante en el
análisis de datos para comprender la distribución de los valores de la variable
Exitosos.

\begin{figure}[htbp]
    \centering
    \includegraphics[width=4.06111in,height=2.68611in]{img/histogramaConCurvaDeDensidad.png}
    \caption{Histograma con Curva de Densidad}
    \label{fig:hist_density}
\end{figure}

La figura \ref{fig:hist_density} muestra este tipo de gráfico correspondiente a
los datos analizados.

El eje Y muestra la densidad en el rango de 0.00 a aproximadamente 0.08,
mientras que el eje X representa la cantidad de éxitos de 0 a 25. La mayor
concentración de la distribución se encuentra alrededor de 0.08 en el eje Y y 0
en el eje X, lo cual indica que la mayoría de las observaciones tienen un
número bajo de éxitos.

La curva de densidad, en color rojo, comienza cerca de (0.02, 0) en el eje Y y
se extiende hasta (10, 0.04) en el eje X. Esta curva suavizada muestra la forma
general de la distribución de los valores de Exitosos. A medida que aumenta la
cantidad de éxitos, la densidad tiende a incrementar, lo cual se refleja en la
curva que sube desde un punto cercano a 0 en el eje Y hasta aproximadamente
0.04 en el eje X.

En resumen, el análisis del histograma con curva de densidad revela que la
mayoría de las observaciones tienen un número bajo de éxitos, con una
concentración máxima alrededor de 0.08 en el eje Y y 0 en el eje X. Además, la
curva de densidad muestra una tendencia creciente a medida que aumenta la
cantidad de éxitos, alcanzando un pico cerca de 0.04 en el eje X.

\subsection{Identificar valores atípicos}

En el análisis de datos, la detección de valores atípicos es crucial para
identificar observaciones que difieren significativamente de la tendencia
general. Estos valores pueden tener un impacto significativo y requerir
atención especial.

A continuación se muestra una tabla con los valores atípicos identificados
mediante el método del Z-score. Se calculó el Z-score para cada observación
utilizando un umbral de 3 desviaciones estándar. Los valores que superan este
umbral se consideran atípicos y se presentan en la tabla:

\begin{table}[htbp]
    \centering
    \caption{Valores Atípicos}
    \begin{tabular}{cccccc}
        \hline
        \textbf{exitosos} & \textbf{fallidos} & \textbf{envios} & \textbf{sol1} & \textbf{aprobado} \\
        \hline
        156               & 6                 & 41              & 47            & 1.5               & 0                 \\
        163               & 0                 & 47              & 47            & 1.6               & 0                 \\
        164               & 9                 & 38              & 47            & 1.6               & 0                 \\
        249               & 9                 & 38              & 47            & 2.4               & 0                 \\
        262               & 26                & 37              & 63            & 2.5               & 0                 \\
        370               & 26                & 5               & 31            & 3.4               & 0                 \\
        514               & 7                 & 40              & 47            & 4.3               & 1                 \\
        523               & 28                & 35              & 63            & 4.4               & 1                 \\
        542               & 26                & 5               & 31            & 4.6               & 1                 \\
        829               & 5                 & 42              & 47            & 7.0               & 1                 \\
        \hline
    \end{tabular}
    \label{tab:valores_atipicos}
\end{table}

Observando los valores atípicos en la tabla \ref{tab:valores_atipicos}, podemos
notar que algunas observaciones difieren significativamente en al menos una de
las variables. Exitosos representa las respuestas correctas de la guía, 
Fallidos indica la cantidad de errores para lograr los Exitosos, y Envíos es la suma de Exitosos y Fallidos. 
Sol1 se refiere a la nota obtenida en la solemne 1. Estos valores atípicos pueden ser de interés para un análisis más detallado, 
ya que podrían indicar situaciones excepcionales o errores en la recolección de datos.