\hypertarget{problema}{%
    \section{Problema}\label{problema}}

El problema de esta investigación se centra en determinar la relación entre
la resolución de una guía de programación y el éxito académico en el ramo
de introducción a la programación de la Universidad Andrés Bello. Se busca responder
a las siguientes preguntas: ¿La resolución de la guía es significativa para aprobar el ramo?
¿La guía puede ser una herramienta predictiva de deserción en la carrera?

Además, se plantea la necesidad de analizar la relevancia de la guía en la preparación de
los estudiantes para la primera prueba o solemne 1 del ramo, evaluando si la resolución de
los ejercicios de la guía proporciona una base sólida para afrontar exitosamente la evaluación.

Para abordar este problema, se utilizará un conjunto de datos llamado "dataset a 2021",
que contiene información sobre los resultados de la guía y el rendimiento en la
solemne 1 del año 2021. Se realizará una revisión bibliográfica de los últimos 10 años p
ara recopilar información relevante y se aplicará el enfoque predictivo de redes bayesianas
para analizar la relación entre la resolución de la guía y el éxito académico,
así como su posible utilidad como predictor de deserción.

El estudio de este problema tiene como objetivo proporcionar una comprensión más precisa
sobre la importancia de la guía de programación y su influencia en el rendimiento académico de los estudiantes.
Los resultados obtenidos permitirán mejorar las estrategias de apoyo a los estudiantes y la toma de decisiones
por parte de la universidad en relación con la implementación y promoción de la guía como recurso educativo
en el ramo de introducción a la programación.
