\hypertarget{metodología}{%
  \section{Metodología}\label{Metodología}}

\subsection{Enfoque de la investigación}

En esta investigación se empleará un enfoque cuantitativo, con el objetivo de analizar de manera objetiva y cuantificable
la relación entre la resolución de la guía de programación y el éxito académico en el curso de Introducción a la
Programación de la Universidad Andrés Bello.

\subsection{Diseño de investigación}

En este estudio, se empleará la metodología KDD (Knowledge Discovery in Databases, descubrimiento de conocimiento en bases de datos) para llevar a cabo el análisis de los datos. El proceso de KDD consta de varias etapas fundamentales que nos permitirán obtener conocimientos relevantes a partir de los datos recopilados. Estas etapas incluyen:

\textbf{Selección de datos}: En esta etapa, se identificarán y seleccionarán los datos relevantes para el estudio. En nuestro caso, utilizaremos el conjunto de datos "dataset a 2021" que contiene información sobre los resultados de la guía de programación y el rendimiento en la primera evaluación del curso de Introducción a la Programación en la Universidad Andrés Bello en el año 2021.

\textbf{Preparación de datos}: En esta etapa, se realizarán las transformaciones necesarias en los datos seleccionados para garantizar su calidad y adecuación al análisis. Esto puede incluir la limpieza de datos, la eliminación de valores atípicos o faltantes, y la normalización de variables, entre otros procesos.

\textbf{Minería de datos}: En esta etapa, se aplicarán técnicas de minería de datos para descubrir patrones, relaciones y tendencias ocultas en los datos. Utilizaremos técnicas estadísticas y algoritmos de aprendizaje automático para explorar la relación entre la resolución de la guía de programación, el éxito académico y el programa de estudio de los estudiantes.

\textbf{Evaluación de resultados}: En esta etapa, se evaluarán los resultados obtenidos a través de la minería de datos. Se analizarán los patrones identificados, se medirá su significancia estadística y se evaluará su relevancia para los objetivos de la investigación.

\textbf{Interpretación de hallazgos}: Finalmente, en esta etapa, se interpretarán los hallazgos obtenidos a partir del análisis de los datos. Se examinarán los resultados en el contexto de la pregunta de investigación planteada y se realizarán inferencias y conclusiones basadas en los patrones y relaciones descubiertos.

\begin{figure}[H]
  \centering
  \includegraphics[width=4.06111in,height=2.68611in]{img/KDD.png}
  \caption{Flujo gráfico KDD}
  \label{fig:flujo_kdd}
\end{figure}

Al seguir la metodología KDD, nos aseguraremos de seguir un enfoque sistemático y riguroso para el análisis de los datos recopilados, lo que nos permitirá obtener conocimientos significativos y relevantes relacionados con la resolución de la guía de programación, el éxito académico y la deserción estudiantil en el curso de Introducción a la Programación en la Universidad Andrés Bello.


\subsection{Descripción de la base de datos}

En esta investigación, se utiliza un conjunto de datos que registra a los estudiantes que tomaron el
curso de Introducción a la Programación en la Universidad Andrés Bello durante el año 2021.
Estos datos incluyen información sobre el rendimiento de los estudiantes en la resolución de la guía de apoyo para la primera evaluación.

La base de datos cuenta con un total de 839 registros que registran información detallada de los
estudiantes, y se compone de 75 columnas que contienen diversas variables relacionadas con el curso
y el desempeño de los estudiantes.

\begin{table}[H]
  \centering
  \caption{Descripción de variables}
  \begin{tabular}{|l|p{0.6\linewidth}|}
    \hline
    \textbf{Variable}        & \textbf{Descripción}                                                                                                                                                           \\
    \hline
    sol1                     & Indica la calificación obtenida en la primera evaluación, con un valor binario donde 1 indica una respuesta correcta a la pregunta, mientras que el valor predeterminado es 0. \\
    \hline
    exitosos                 & Indica la cantidad de preguntas respondidas correctamente en la guía.                                                                                                          \\
    \hline
    fallidos                 & Indica la cantidad de preguntas respondidas incorrectamente en la guía.                                                                                                        \\
    \hline
    hito1                    & Son las espectativas a cumplir del aprendizaje del curso.                                                                                                                      \\
    \hline
    hito2                    & Son las espectativas a cumplir del aprendizaje del curso.                                                                                                                      \\
    \hline
    programa                 & Indica al programa de estudio.                                                                                                                                                 \\
    \hline
    Columnas e0 hasta la e52 & Son representacion de los resultados de las preguntas de la guia este es binario.                                                                                              \\
    \hline
  \end{tabular}
  \label{tab:variables}
\end{table}


Estas columnas son relevantes para nuestro análisis, ya que nos permitirán examinar la relación
entre la resolución de la guía de programación, el éxito académico en la primera evaluación y
el programa de estudio al que pertenecen los estudiantes (véase Tabla \ref{tab:variables}).



\subsection{Recopilación de datos}

Para llevar a cabo este estudio, se cuenta con el conjunto de datos denominado "dataset a 2021", que contiene la información necesaria sobre los
resultados de la guía de programación y el rendimiento en la primera evaluación del curso de Introducción a la Programación en el año 2021.


\subsection{Análisis de datos}

Una vez recopilados los datos, se realizará un análisis descriptivo para examinar la distribución de los resultados en la guía de programación y la primera evaluación. Además, se llevará a cabo un análisis de correlación entre las variables mencionadas anteriormente para identificar posibles
relaciones y patrones significativos.

\subsection{Aplicación de XAI y SHAP}

En esta investigación, se planea utilizar técnicas de XAI (Explicabilidad de la Inteligencia Artificial) y SHAP (Shapley Additive Explanations) para comprender mejor el impacto de la resolución de la guía de programación en el éxito académico en el curso de Introducción a la Programación de la Universidad Andrés Bello.

XAI proporciona métodos y herramientas para interpretar y explicar las decisiones tomadas por los modelos de inteligencia artificial. En este estudio, se utilizará XAI para analizar los factores que influyen en la relación entre la resolución de la guía de programación y el éxito académico. Esto permitirá identificar las características o patrones específicos en las respuestas de los estudiantes que están relacionados con un mayor rendimiento académico.

Además, se empleará SHAP como una técnica de atribución para evaluar la importancia relativa de las diferentes características de la guía de programación en la predicción del éxito académico. SHAP proporciona una medida cuantitativa de la contribución de cada variable en la toma de decisiones del modelo. Mediante el uso de SHAP, se podrá determinar qué aspectos de la resolución de la guía de programación tienen un mayor impacto en el rendimiento académico de los estudiantes.

Al combinar el enfoque cuantitativo con XAI y SHAP, esta investigación pretende brindar una comprensión más profunda y detallada de la relación entre la resolución de la guía de programación y el éxito académico. Estas técnicas permitirán identificar los aspectos críticos que pueden influir en el rendimiento de los estudiantes y proporcionar información valiosa para mejorar los procesos de enseñanza y aprendizaje en el curso de Introducción a la Programación.