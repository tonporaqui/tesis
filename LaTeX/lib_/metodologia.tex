\hypertarget{metodología}{%
    \section{Metodología}\label{Metodología}}
\vfill
\subsection*{Enfoque de la investigación}

En esta investigación se empleará un enfoque cuantitativo, con el objetivo de analizar de manera objetiva y cuantificable
la relación entre la resolución de la guía de programación y el éxito académico en el curso de Introducción a la
Programación de la Universidad Andrés Bello.
\vfill
\subsection*{Diseño de investigación}

Se utilizará la metodología KDD (Knowledge Discovery in Databases, por sus siglas en inglés) para realizar el análisis de los datos.
El proceso KDD comprende una serie de etapas, como la selección y preparación de los datos, la transformación y minería de datos,
la evaluación de los resultados y la interpretación de los hallazgos.

\vfill
\subsection*{Descripción de la base de datos}

El conjunto de datos utilizado en esta investigación se compone del registro de los estudiantes que tomaron el curso de
Introducción a la Programación en el año 2021 en la Universidad Andrés Bello. Estos datos incluyen información relacionada con la
resolución de la guía de apoyo para la primera evaluación.

La base de datos cuenta con un total de 839 registros y se compone de 75 columnas. Para el análisis,
se han seleccionado las siguientes columnas relevantes:

\begin{center}
    \begin{tabular}{|l|p{0.6\linewidth}|}
      \hline
      \textbf{Variable} & \textbf{Descripción} \\
      \hline
      sol1 & Representa la calificación obtenida en la primera evaluación, con un valor binario donde 1 indica una respuesta correcta a la pregunta, mientras que el valor predeterminado es 0. \\
      \hline
      exitosos & Representa la cantidad de preguntas respondidas correctamente en la guía. \\
      \hline
      fallidos & Representa la cantidad de preguntas respondidas incorrectamente en la guía. \\
      \hline
      envíos & Representa la suma de respuestas exitosas y fallidas. \\
      \hline
      programa & Representa al programa de estudio. \\
      \hline
    \end{tabular}
  \end{center}

\vfill
Estas columnas son relevantes para nuestro análisis, ya que nos permitirán examinar la relación entre la resolución de la guía de programación,
el éxito académico en la primera evaluación y el programa de estudio al que pertenecen los estudiantes.

\vfill
\subsection*{Recopilación de datos}

Para llevar a cabo este estudio, se obtendrá el conjunto de datos denominado "dataset a 2021", que contiene la información necesaria sobre los
resultados de la guía de programación y el rendimiento en la primera evaluación del curso de Introducción a la Programación en el año 2021.

\vfill
\subsection*{Análisis de datos}

Una vez recopilados los datos, se realizará un análisis descriptivo para examinar la distribución de los resultados en la guía de programación y
la primera evaluación. Además, se llevará a cabo un análisis de correlación entre las variables mencionadas anteriormente para identificar posibles
relaciones y patrones significativos.
\vfill