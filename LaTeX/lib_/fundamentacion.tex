\hypertarget{Fundamentación}{%
\section{Fundamentación}\label{Fundamentación}}

La educación superior en informática y programación trasciende las demandas del mercado laboral, reflejando cómo la tecnología está transformando nuestra sociedad. En este escenario, la programación no solo se presenta como una habilidad técnica, sino también como un enfoque de pensamiento y resolución de problemas.

La educación superior, particularmente en campos técnicos como la informática, es crucial para cultivar profesionales capaces de navegar y adaptarse a las vertiginosas transformaciones tecnológicas. Hair, Black, Babin, Anderson, \& Tatham (2019) \cite{hair2019advanced} enfatizan que la educación superior es fundamental para formar individuos equipados para los retos del mundo actual. No obstante, debido a su naturaleza abstracta y lógica, la programación puede representar un desafío considerable para muchos estudiantes.

Las guías de programación sirven como nexos entre la teoría y la aplicación. Han, Kamber, \& Pei (2011) \cite{han2011data} y García, Romero, \& Ventura (2018) \cite{garcia2018prediccion} sostienen que, al proporcionar ejercicios prácticos y contextualizados, estas guías son vitales para consolidar el aprendizaje teórico y práctico, preparando a los estudiantes para evaluaciones y situaciones reales.

En una era dominada por la data, las técnicas convencionales de análisis ya no bastan. Las redes bayesianas, descritas por Koller y Friedman (2009) \cite{koller2009introduction}, ofrecen una visión más enriquecida, permitiendo modelar la incertidumbre y las interacciones entre variables, lo que es crucial para entender las dinámicas inherentes al aprendizaje.

La interpretabilidad de los modelos ha cobrado importancia en tiempos recientes. La demanda de modelos claros y comprensibles es primordial, sobre todo en el contexto educativo. Lipton (2018) \cite{lipton2018mythos} y Doshi-Velez \& Kim (2017) \cite{doshivelez2017rigorous} postulan que confiar en los modelos es indispensable para asegurar decisiones equitativas y éticas. Métodos como SHAP y XAI, introducidos por Lundberg y Lee (2017) \cite{lundberg2017unified}, se han establecido como referentes en el ámbito de la explicabilidad.

La inferencia causal supera el análisis meramente correlacional, apuntando a discernir relaciones causales entre variables. Pearl (2009) \cite{pearl2009introduction} y Geffner et al. (2022) \cite{geffner2022deep} han resaltado la relevancia de este enfoque. Herramientas innovadoras como DoWhy (Sharma \& Kiciman, 2020) \cite{sharma2020dowhy} y DoWhy-GCM (Blöbaum et al., 2022) \cite{blobaum2022dowhy} han emergido para ofrecer marcos robustos que facilitan el análisis de las causas raíz del rendimiento académico y otros fenómenos.

En resumen, la educación superior en programación es un dominio en constante metamorfosis que demanda una perspectiva integrada y diversa. La amalgama de guías de programación, técnicas avanzadas de análisis de datos, explicabilidad y herramientas de inferencia causal configura un marco robusto y detallado para enfrentar los desafíos inherentes a la enseñanza y el aprendizaje en programación.
