\hypertarget{Fundamentación}{%
    \section{Fundamentación}\label{Fundamentación}}
La educación desempeña un papel fundamental en el desarrollo de las personas y las sociedades. En particular, la educación superior tiene como objetivo formar a los futuros profesionales y fomentar la generación de conocimiento e innovación. Sin embargo, el éxito académico en la educación superior puede ser un desafío debido a diversos factores, como la complejidad de los contenidos y la falta de experiencia previa en programación. Para ayudar a los estudiantes a superar estos desafíos y mejorar su rendimiento académico, es importante contar con herramientas y estrategias efectivas.

En este contexto, las guías de programación se presentan como una herramienta valiosa para apoyar el aprendizaje de los estudiantes en el ámbito de la introducción a la programación. Estas guías proporcionan ejercicios y problemas que permiten a los estudiantes practicar y aplicar los conceptos teóricos aprendidos en clase. Además, las guías pueden servir como una forma efectiva de prepararse para las evaluaciones y mejorar el rendimiento académico en esta área \cite{han2011data, garcia2018prediccion}.

Sin embargo, es importante analizar si la resolución de una guía de programación está realmente relacionada con el éxito académico en el ramo de introducción a la programación. Para abordar esta cuestión, se propone utilizar un enfoque basado en el marco unificado SHAP (SHapley Additive exPlanations) y XAI (Explicabilidad de la Inteligencia Artificial) \cite{lundberg2017unified, ribeiro2016trust, doshivelez2017rigorous}. Estos enfoques proporcionan herramientas para analizar y comprender el impacto de diferentes variables y características en la resolución de las guías de programación, y su relación con el éxito académico en el ramo de introducción a la programación.

El uso de SHAP y XAI en este estudio permitirá generar explicaciones claras y comprensibles sobre cómo la resolución de las guías de programación puede influir en el éxito académico de los estudiantes. Estas explicaciones ayudarán a mejorar las estrategias de apoyo a los estudiantes y respaldarán la toma de decisiones informadas en relación con el uso de las guías como recurso educativo \cite{lipton2018mythos}.

En resumen, este estudio tiene como objetivo investigar la relación entre la resolución de las guías de programación y el éxito académico en el ramo de introducción a la programación. Para ello, se utilizará el marco unificado SHAP y XAI para analizar y comprender cómo las guías de programación influyen en el rendimiento de los estudiantes. Los resultados obtenidos serán de gran utilidad para mejorar las estrategias de apoyo a los estudiantes y la toma de decisiones relacionadas con el uso de las guías como herramienta educativa.
