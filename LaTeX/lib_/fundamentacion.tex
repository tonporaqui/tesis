\hypertarget{Fundamentación}{%
\section{Fundamentación}\label{Fundamentación}}

La educación superior, como señalan \textit{Hair et al.} \cite{hair2019advanced}, es un pilar fundamental en la formación de individuos capaces de enfrentar los desafíos del mundo contemporáneo. Sin embargo, el camino hacia el éxito académico está plagado de obstáculos, y la programación, siendo una disciplina que requiere una combinación de habilidades lógicas y creativas, no es una excepción.

Las guías de programación, respaldadas por investigaciones como las de \textit{Han, Kamber y Pei} \cite{han2011data} y \textit{García, Romero y Ventura} \cite{garcia2018prediccion}, se han consolidado como herramientas esenciales para reforzar el aprendizaje teórico y práctico. Estas guías, al ofrecer ejercicios prácticos y contextualizados, permiten a los estudiantes consolidar sus conocimientos y prepararse adecuadamente para las evaluaciones.

Sin embargo, en la era de la data y la inteligencia artificial, la simple observación y correlación ya no son suficientes. Es aquí donde técnicas avanzadas, como las redes bayesianas descritas por \textit{Koller y Friedman} \cite{koller2009introduction}, ofrecen un enfoque más profundo y matizado para analizar el impacto de diversas variables en el rendimiento académico. Estas herramientas, al modelar la incertidumbre y las relaciones entre variables, proporcionan insights valiosos sobre las dinámicas subyacentes del aprendizaje.

Aun así, la interpretabilidad y confiabilidad de los modelos se ha convertido en un tema de debate en la comunidad científica. Como discuten \textit{Lipton} \cite{lipton2018mythos} y \textit{Doshi-Velez y Kim} \cite{doshivelez2017rigorous}, la transparencia y explicabilidad de los modelos son esenciales para garantizar que las decisiones basadas en ellos sean justas y éticas. En este contexto, SHAP y XAI, explorados por \textit{Lundberg y Lee} \cite{lundberg2017unified}, emergen como técnicas líderes en el campo de la explicabilidad, permitiendo a los educadores y responsables de la toma de decisiones comprender y confiar en las predicciones generadas.

Más allá de la simple correlación, la inferencia causal se presenta como una herramienta esencial para desentrañar las relaciones causales subyacentes. Como señalan \textit{Pearl} \cite{pearl2009introduction} y \textit{Geffner et al.} \cite{geffner2022deep}, entender las causas raíz de un fenómeno permite no solo predecir, sino también actuar sobre él. Herramientas como DoWhy \cite{sharma2020dowhy} y DoWhy-GCM \cite{blobaum2022dowhy} representan avances significativos en este ámbito, proporcionando marcos robustos para analizar las causas subyacentes del rendimiento académico.

En resumen, la educación superior, y en particular la programación, es un campo complejo y multifacético. La combinación de guías de programación, técnicas avanzadas de análisis de datos, explicabilidad y inferencia causal ofrece un enfoque holístico y bien fundamentado para abordar los desafíos inherentes a este campo, con el objetivo final de mejorar el éxito académico y reducir el abandono escolar.