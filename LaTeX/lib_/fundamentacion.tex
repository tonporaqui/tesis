\hypertarget{Fundamentación}{%
\section{Fundamentación}\label{Fundamentación}}

La educación superior en informática y programación no es solo una respuesta a las demandas del mercado laboral, sino también una reflexión de cómo la tecnología está reconfigurando nuestra sociedad. En este contexto, la programación emerge no solo como una habilidad técnica, sino también como una forma de pensamiento y resolución de problemas.

La educación superior, especialmente en áreas técnicas como la informática, es esencial para formar profesionales que puedan enfrentar y adaptarse a los rápidos cambios tecnológicos. \textit{Hair et al.} \cite{hair2019advanced} destacan que la educación superior es un pilar en la formación de individuos preparados para los desafíos del mundo contemporáneo. Sin embargo, la programación, con su naturaleza abstracta y lógica, puede ser un desafío particularmente arduo para muchos estudiantes.

Las guías de programación actúan como puentes entre la teoría y la práctica. Según \textit{Han, Kamber y Pei} \cite{han2011data} y \textit{García, Romero y Ventura} \cite{garcia2018prediccion}, estas guías, al ofrecer ejercicios prácticos y contextualizados, son esenciales para reforzar el aprendizaje teórico y práctico, y para preparar a los estudiantes para evaluaciones y aplicaciones del mundo real.

En la era actual, donde la data es omnipresente, las técnicas tradicionales de análisis ya no son suficientes. Las redes bayesianas, como las que describen \textit{Koller y Friedman} \cite{koller2009introduction}, ofrecen una perspectiva más profunda, permitiendo modelar la incertidumbre y las relaciones entre variables, lo que resulta esencial para comprender las dinámicas subyacentes del aprendizaje.

La interpretabilidad de los modelos es un tema que ha ganado relevancia en los últimos años. La necesidad de modelos transparentes y explicables es fundamental, especialmente en el ámbito educativo. \textit{Lipton} \cite{lipton2018mythos} y \textit{Doshi-Velez y Kim} \cite{doshivelez2017rigorous} argumentan que la confianza en los modelos es esencial para garantizar decisiones justas y éticas. Técnicas como SHAP y XAI, presentadas por \textit{Lundberg y Lee} \cite{lundberg2017unified}, se han consolidado como líderes en el campo de la explicabilidad.

La inferencia causal va más allá del análisis correlacional, buscando identificar relaciones causales entre variables. \textit{Pearl} \cite{pearl2009introduction} y \textit{Geffner et al.} \cite{geffner2022deep} han subrayado la importancia de esta herramienta. Herramientas como DoWhy \cite{sharma2020dowhy} y DoWhy-GCM \cite{blobaum2022dowhy} han surgido para proporcionar marcos robustos que permiten analizar las causas subyacentes del rendimiento académico y otros fenómenos.

Concluyendo, la educación superior en programación es un campo en constante evolución que requiere un enfoque integrado y multifacético. La combinación de guías de programación, técnicas avanzadas de análisis de datos, explicabilidad y herramientas de inferencia causal proporciona un marco sólido y profundo para abordar los desafíos de la enseñanza y el aprendizaje en programación.
