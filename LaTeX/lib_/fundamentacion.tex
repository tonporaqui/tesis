\hypertarget{Fundamentación}{%
    \section{Fundamentación}\label{Fundamentación}}
La educación es un factor clave para el desarrollo de las personas y las sociedades. En este sentido, la educación superior tiene un papel fundamental en la formación de los futuros profesionales y en la generación de conocimiento e innovación. Sin embargo, el éxito académico en la educación superior no siempre es fácil de alcanzar debido a diversos
factores, como la complejidad de los contenidos, la falta de motivación o el estrés académico. Por esta razón, es importante contar con herramientas y estrategias que ayuden a los estudiantes a enfrentar estos desafíos y mejorar su rendimiento académico.

En este contexto, las guías de programación pueden ser una herramienta útil para apoyar el aprendizaje de los estudiantes en el ramo de introducción a la programación. Estas guías contienen ejercicios y problemas que permiten a los estudiantes practicar y aplicar los conceptos teóricos aprendidos en clase. Además, pueden ser una forma efectiva de
prepararse para las evaluaciones y mejorar el rendimiento académico.

Por otro lado, las redes bayesianas son una técnica avanzada de análisis de datos que permite modelar relaciones causales entre variables y generar predicciones sobre eventos futuros. En el contexto educativo, las redes bayesianas pueden ser útiles para analizar la relevancia de diferentes factores en el éxito académico y predecir la deserción estudiantil.

En este trabajo se propone utilizar redes bayesianas para analizar si la resolución de una guía de programación está relacionada con el éxito académico en el ramo de introducción a la programación en la Universidad Andrés Bello. Se espera que los resultados de este estudio puedan ser útiles para mejorar las estrategias de apoyo a los estudiantes y la toma de decisiones informadas en relación con las guías como recurso educativo.