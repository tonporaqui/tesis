\hypertarget{discusiuxf3n}{%
\section{Discusión}\label{discusiuxf3n}}

La educación, como bien señalan \textit{Hair et al.} \cite{hair2019advanced}, no es solo un proceso de transmisión de conocimientos, sino un complejo entramado donde intervienen múltiples factores que determinan el éxito o fracaso académico de los estudiantes. En este contexto, el rendimiento académico y el abandono escolar se erigen como dos caras de una misma moneda, siendo indicadores cruciales que reflejan la eficacia y eficiencia de los sistemas educativos.

Las redes bayesianas, exploradas en profundidad por \textit{Koller y Friedman} \cite{koller2009introduction}, ofrecen un enfoque probabilístico que permite modelar la incertidumbre y las relaciones entre variables. Estas herramientas son especialmente útiles en contextos donde las relaciones causales no son directamente observables, como es el caso del rendimiento académico. Aunque las redes bayesianas son altamente interpretables, pueden ser complejas. Por lo tanto, antes de sumergirse en la complejidad de estas redes, es conveniente entender la importancia de cada variable respecto a la tarea de interés.

Es en este punto donde herramientas como SHAP, descrito por \textit{Lundberg y Lee} \cite{lundberg2017unified}, se convierten en esenciales. SHAP no solo proporciona un marco para interpretar las decisiones de los modelos de aprendizaje automático, sino que también permite asignar un valor a cada característica en función de su contribución al resultado final. En el ámbito educativo, esto se traduce en la capacidad de identificar qué factores tienen un impacto significativo en el rendimiento académico y, por ende, en la toma de decisiones informadas para diseñar intervenciones pedagógicas.

A pesar de la eficacia de técnicas como los bosques aleatorios \cite{breiman2001random} y el aumento de gradiente \cite{friedman2000additive} en la predicción del rendimiento académico, es crucial no confundir correlación con causalidad. La inferencia causal, como discuten \textit{Pearl} \cite{pearl2009introduction} y \textit{Geffner et al.} \cite{geffner2022deep}, busca desentrañar las relaciones causales subyacentes, permitiendo no solo predecir, sino también entender y actuar sobre las causas raíz de los fenómenos observados. Herramientas como DoWhy \cite{sharma2020dowhy} y DoWhy-GCM \cite{blobaum2022dowhy} representan avances revolucionarios en este ámbito, proporcionando marcos robustos para analizar las causas subyacentes del rendimiento académico y el abandono escolar.

Finalmente, es esencial reconocer que la educación es un sistema dinámico, influenciado por una multitud de factores tanto internos como externos. La combinación de técnicas avanzadas de análisis de datos, modelos gráficos, explicabilidad y inferencia causal proporciona un enfoque holístico y multidimensional para abordar estos desafíos. Estas herramientas, respaldadas por investigaciones de \textit{García et al.} \cite{garcia2018prediccion}, \textit{Wang et al.} \cite{wang2017literature}, y otros, son esenciales para informar y guiar estrategias educativas efectivas, con el objetivo final de mejorar el éxito estudiantil y reducir el abandono escolar.