\hypertarget{discusiuxf3n}{%
    \section{Discusión}\label{discusiuxf3n}}
    \vfill
La aplicación de modelos gráficos y el enfoque bayesiano en el análisis de datos ha demostrado ser una herramienta poderosa
en la modelización de sistemas complejos (Koller \& Friedman, 2009 \cite{koller2009introduction}). Estos modelos proporcionan
una forma intuitiva de representar las relaciones entre variables y permiten el razonamiento probabilístico mediante el uso de reglas de inferencia bayesianas.

En el campo del análisis de datos, el uso de técnicas avanzadas ha permitido extraer información valiosa y conocimiento profundo de conjuntos
de datos complejos (Hair et al., 2019 \cite{hair2019advanced}). Estas técnicas incluyen métodos estadísticos y algoritmos de aprendizaje
automático que ayudan a descubrir patrones, identificar relaciones causales y realizar predicciones precisas.

En el razonamiento de redes bayesianas con evidencia incierta, se busca manejar la incertidumbre presente en los datos observados y
actualizar las creencias sobre las variables de interés (Jensen, 2001 \cite{jensen2001bayesian}). Las redes bayesianas proporcionan
una forma de representar y actualizar el conocimiento incierto mediante la propagación de creencias utilizando reglas de inferencia probabilística.

El análisis de datos ha evolucionado desde la minería de datos hacia enfoques más sofisticados y avanzados (Han et al., 2011 \cite{han2011data}).
Ahora se enfoca en explorar relaciones causales y descubrir conocimiento más profundo a partir de los datos.

La inferencia causal utilizando redes bayesianas y el cálculo do ha surgido como una herramienta poderosa para comprender las relaciones
causales en conjuntos de datos observacionales (Pearl, 2009 \cite{pearl2009introduction}). Estos modelos permiten identificar las relaciones
causales subyacentes y realizar inferencias causales a partir de datos observacionales.

La predicción del abandono escolar es un tema de investigación activo y de gran relevancia en el campo educativo (Wang et al., 2017 \cite{wang2017literature}).
La aplicación de técnicas de minería de datos ha demostrado ser útil para identificar a los estudiantes en riesgo y proporcionar intervenciones tempranas.

Es fundamental tener modelos interpretables para comprender y confiar en las predicciones generadas por los algoritmos de aprendizaje automático
(Kocev et al., 2013 \cite{kocev2013need}). Estos modelos permiten a los responsables de la toma de decisiones comprender los resultados y tomar
acciones adecuadas.

La predicción del rendimiento académico mediante técnicas de minería de datos ha demostrado ser un enfoque prometedor
(García et al., 2018 \cite{garcia2018prediccion}). La aplicación de estas técnicas permite identificar patrones y tendencias en los datos académicos,
lo que puede ayudar a predecir el rendimiento futuro de los estudiantes.

En resumen, el uso de modelos gráficos, el enfoque bayesiano, técnicas avanzadas de análisis de datos y minería de datos ofrece una perspectiva integral para
comprender y predecir el rendimiento académico y el abandono escolar (Koller \& Friedman, 2009 \cite{koller2009introduction};
Hair et al., 2019 \cite{hair2019advanced}; Jensen, 2001 \cite{jensen2001bayesian}; Han et al., 2011 \cite{han2011data}; Pearl, 2009 \cite{pearl2009introduction};
Wang et al., 2017 \cite{wang2017literature}; Kocev et al., 2013 \cite{kocev2013need}; García et al., 2018 \cite{garcia2018prediccion}).
Estas herramientas proporcionan una base sólida para la toma de decisiones informadas en el ámbito educativo y la implementación de intervenciones tempranas
para mejorar el éxito estudiantil.

Los modelos gráficos y las redes bayesianas permiten capturar las complejas relaciones entre las variables académicas. Además, 
estas técnicas proporcionan un marco flexible y probabilístico para modelar el rendimiento académico, realizar inferencias causales y 
analizar la incertidumbre presente en los datos \cite{jensen2001bayesian}.

La aplicación de técnicas avanzadas de análisis de datos y minería de datos, como algoritmos de aprendizaje automático, ha permitido descubrir patrones ocultos
y relaciones causales en conjuntos de datos complejos \cite{hair2019advanced, han2011data}. Estas técnicas ayudan a identificar factores predictivos y
proporcionan modelos de predicción precisos para predecir el rendimiento académico y el abandono escolar.

Es importante destacar la necesidad de modelos interpretables que permitan comprender y confiar en las predicciones generadas \cite{kocev2013need}.
La interpretabilidad es crucial en el contexto educativo, ya que los responsables de la toma de decisiones deben comprender los factores que influyen
en el rendimiento académico y tomar acciones adecuadas para mejorar los resultados de los estudiantes.

La predicción del rendimiento académico mediante técnicas de minería de datos ofrece un enfoque prometedor para identificar a los estudiantes en riesgo
y proporcionar intervenciones tempranas \cite{garcia2018prediccion, wang2017literature}. Al analizar datos académicos históricos y otros factores relevantes,
se pueden desarrollar modelos predictivos que ayuden a identificar a los estudiantes que pueden necesitar apoyo adicional.

En conclusión, la combinación de modelos gráficos, enfoques bayesianos, técnicas avanzadas de análisis de datos y minería de datos ofrece un enfoque poderoso
para comprender y predecir el rendimiento académico y el abandono escolar. Estas herramientas proporcionan a educadores y responsables de políticas una base
sólida para la toma de decisiones informadas y la implementación de estrategias efectivas para mejorar el éxito estudiantil.
\vfill