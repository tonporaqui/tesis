\hypertarget{discusiuxf3n}{%
\section{Discusión}\label{discusiuxf3n}}

La educación, más allá de ser un mero proceso de transmisión de conocimientos, es un complejo sistema influenciado por múltiples factores que determinan el éxito o fracaso académico de los estudiantes (Hair, Black, Babin, Anderson, \& Tatham, 2019) \cite{hair2019advanced}. En este panorama, el rendimiento académico y el abandono escolar se presentan como indicadores cruciales que reflejan la eficacia y eficiencia de los sistemas educativos.

Las redes bayesianas, como detallan Koller y Friedman (2009) \cite{koller2009introduction}, ofrecen un enfoque probabilístico para modelar la incertidumbre y las relaciones entre variables. Estas herramientas son especialmente valiosas en contextos donde las relaciones causales no son directamente observables, como en el caso del rendimiento académico. A pesar de su alta interpretabilidad, las redes bayesianas pueden ser intrincadas, lo que subraya la necesidad de comprender la relevancia de cada variable antes de adentrarse en su complejidad.

En este escenario, herramientas como SHAP, propuestas por Lundberg y Lee (2017) \cite{lundberg2017unified}, se vuelven fundamentales. SHAP no solo brinda un marco para interpretar las decisiones de los modelos de aprendizaje automático, sino que también asigna un valor a cada característica basado en su contribución al resultado. En el contexto educativo, esto se traduce en identificar factores que impactan significativamente en el rendimiento académico, facilitando la toma de decisiones pedagógicas informadas.

Aunque técnicas como los bosques aleatorios (Breiman, 2001) \cite{breiman2001random} y el aumento de gradiente (Friedman, Hastie, \& Tibshirani, 2000) \cite{friedman2000additive} han demostrado ser eficaces en la predicción del rendimiento académico, es vital distinguir entre correlación y causalidad. La inferencia causal, discutida por Pearl (2009) \cite{pearl2009introduction} y Geffner et al. (2022) \cite{geffner2022deep}, se centra en desentrañar las relaciones causales, permitiendo no solo predecir, sino también comprender y actuar sobre las causas fundamentales de los fenómenos observados. Herramientas innovadoras como DoWhy (Sharma \& Kiciman, 2020) \cite{sharma2020dowhy} y DoWhy-GCM (Blöbaum et al., 2022) \cite{blobaum2022dowhy} ofrecen marcos robustos para analizar las causas subyacentes del rendimiento académico y el abandono escolar.

En conclusión, la educación es un sistema dinámico, influenciado por diversos factores tanto internos como externos. La combinación de técnicas avanzadas de análisis de datos, modelos gráficos, explicabilidad e inferencia causal proporciona un enfoque integral para enfrentar estos desafíos. Estas herramientas, respaldadas por investigaciones como las de García, Romero, \& Ventura (2018) \cite{garcia2018prediccion} y Wang, Yang, Zhang, \& Zhou (2017) \cite{wang2017literature}, son cruciales para informar y guiar estrategias educativas con el objetivo de mejorar el éxito estudiantil y reducir el abandono escolar.
