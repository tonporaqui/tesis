\hypertarget{discusiuxf3n}{%
    \section{Discusión}\label{discusiuxf3n}}

    La combinación de modelos gráficos, técnicas avanzadas de análisis de datos y minería de datos ofrece un enfoque poderoso para comprender y predecir el rendimiento académico y el abandono escolar \cite{hair2019advanced}. Estas herramientas proporcionan una base sólida para la toma de decisiones informadas y la implementación de estrategias efectivas para mejorar el éxito estudiantil \cite{garcia2018prediccion, wang2017literature}.

    La aplicación de modelos gráficos, en particular las redes bayesianas, ha demostrado ser una herramienta poderosa en el análisis de datos y la modelización de sistemas complejos \cite{koller2009introduction}. Las redes bayesianas proporcionan una forma intuitiva de representar las relaciones entre variables y permiten el razonamiento probabilístico mediante el uso de reglas de inferencia bayesianas \cite{koller2009introduction, jensen2001bayesian}. Estos modelos son especialmente útiles en el manejo de la incertidumbre presente en los datos observados y en la actualización de las creencias sobre las variables de interés \cite{jensen2001bayesian}.
    
    Sin embargo, si deseamos enfocarnos en la interpretabilidad y la predicción de los datos, podemos utilizar enfoques más específicos. Por ejemplo, el uso de técnicas de explicabilidad de modelos, como XAI (Explicabilidad de Inteligencia Artificial), puede brindar una comprensión más clara de cómo los algoritmos de aprendizaje automático toman decisiones y generan predicciones \cite{lipton2018mythos, ribeiro2016trust}. Estas técnicas permiten a los responsables de la toma de decisiones comprender y confiar en las predicciones generadas.
    
    El enfoque SHAP (Shapley Additive Explanations) es una técnica prometedora en XAI que proporciona una explicación global y local de las predicciones de un modelo \cite{lundberg2017unified}. SHAP asigna valores de importancia a cada característica en función de su contribución al resultado final, lo que permite una mayor transparencia en el proceso de toma de decisiones.
    
    La interpretabilidad de los modelos es fundamental en el contexto educativo, ya que los responsables de la toma de decisiones deben comprender los factores que influyen en el rendimiento académico y tomar acciones adecuadas para mejorar los resultados de los estudiantes \cite{kocev2013need}. La capacidad de interpretar y confiar en los resultados de los modelos de aprendizaje automático es esencial para que los educadores implementen intervenciones tempranas y estrategias efectivas \cite{doshivelez2017rigorous}.
    
    Por otro lado, la predicción del rendimiento académico y el abandono escolar es un tema de investigación activo y relevante en el campo educativo \cite{garcia2018prediccion, wang2017literature}. La aplicación de técnicas de minería de datos y aprendizaje automático ha demostrado ser útil para identificar a los estudiantes en riesgo y proporcionar intervenciones tempranas \cite{garcia2018prediccion, han2011data}.
    
    Técnicas como los bosques aleatorios \cite{breiman2001random} y el aumento de gradiente (boosting) \cite{friedman2000additive} se han utilizado ampliamente en la predicción del rendimiento académico. Los bosques aleatorios son un método de aprendizaje automático que construye una colección de árboles de decisión para realizar predicciones precisas \cite{breiman2001random}. Por otro lado, el aumento de gradiente es una técnica que construye un modelo de predicción en etapas sucesivas, cada vez ajustándose a los errores cometidos por el modelo anterior \cite{friedman2000additive}.
    
    En conclusión, la combinación de modelos gráficos, técnicas avanzadas de análisis de datos y explicabilidad utilizando SHAP ofrece un enfoque poderoso para comprender y predecir el rendimiento académico y el abandono escolar \cite{garcia2018prediccion, wang2017literature, lundberg2017unified}. Estas herramientas proporcionan una base sólida para la toma de decisiones informadas en el ámbito educativo y la implementación de estrategias efectivas para mejorar el éxito estudiantil. Además, las técnicas de interpretabilidad y los enfoques de minería de datos como los bosques aleatorios y el aumento de gradiente ofrecen vías prometedoras para abordar estos desafíos \cite{kocev2013need, breiman2001random, friedman2000additive}.