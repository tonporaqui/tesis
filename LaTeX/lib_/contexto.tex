\hypertarget{contexto}{%
    \section{Contexto}\label{contexto}}
Este estudio examina si la resolución de una guía de programación está relacionada con el éxito
académico en el ramo de introducción a la programación de la Universidad Andrés Bello. Se investiga
si la guía, que consta de 52 ejercicios y no es obligatoria, ayuda a los estudiantes a enfrentar la
primera prueba del ramo. Además, se analiza si la guía puede predecir la deserción en la carrera.

Se realiza una revisión bibliográfica de los últimos 10 años para recopilar estudios relacionados.
Se propone el uso de redes bayesianas para generar predicciones sobre la relevancia de la guía
en la aprobación del ramo y su capacidad predictiva para la deserción.

Los resultados contribuirán a comprender mejor la influencia de la guía de programación
en el éxito académico de los estudiantes. Además, se espera que los hallazgos mejoren
las estrategias de apoyo a los estudiantes y la toma de decisiones de la universidad
en relación con la guía como recurso educativo.

Palabras clave: guía de programación, éxito académico, redes bayesianas, predicción, deserción, introducción a la programación.