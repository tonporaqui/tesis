\begin{center}
    \textbf{\LARGE \textit{Título}}
    \vfill
    Gastón Ernesto Sepulveda Espinoza

    Bajo la supervisión del Profesor Billy Mark Peralta Márquez en la Universidad Andrés Bello

\end{center}
\vfill

\begin{center}
    \textbf{\LARGE \textit{Abstract}}
\end{center}

This paper examines whether the completion of a programming guide is related to academic success in the introductory programming course at Universidad Andrés Bello. It investigates whether the guide, consisting of 52 exercises, which is not mandatory to solve, helps students prepare for the first test of the course and if it can predict dropout rates in the program.
Additionally, the use of the unified SHAP and XAI framework is proposed to analyze and understand how programming guides influence students' performance. The study includes a literature review on related previous studies, a descriptive analysis of the collected data, and a discussion on the practical and theoretical implications of the study.
The obtained results will be useful for improving support strategies for students and decision-making related to the use of guides as an educational tool.

\vfill

\begin{center}
    \textbf{\LARGE \textit{Resumen}}
\end{center}

En este trabajo se examina si la resolución de una guía de programación está relacionada con el éxito académico en el ramo de introducción a la programación en la Universidad Andrés Bello. Se investiga si la guía, que consta de 52 ejercicios la cual no es obligatoria resolver, ayuda a los estudiantes a enfrentar la primera prueba del ramo y si puede predecir la deserción en la carrera.
Además, se propone el uso del marco unificado SHAP y XAI para analizar y comprender cómo las guías de programación influyen en el rendimiento de los estudiantes. El estudio incluye una revisión bibliográfica sobre estudios previos relacionados, un análisis descriptivo de los datos recolectados y una discusión sobre las implicaciones prácticas y teóricas del estudio.
Los resultados obtenidos serán útiles para mejorar las estrategias de apoyo a los estudiantes y la toma de decisiones relacionadas con el uso de las guías como herramienta educativa.
\vfill