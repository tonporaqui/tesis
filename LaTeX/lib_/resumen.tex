\begin{center}
    \textbf{\LARGE \textit{Título}}
    \vfill
    Gastón Ernesto Sepulveda Espinoza

    Bajo la supervisión del Profesor Billy Mark Peralta Márquez en la Universidad Andrés Bello

\end{center}
\vfill
\begin{center}
    \textbf{\LARGE \textit{Resumen}}
\end{center}

En este trabajo se examina si la resolución de una guía de programación está relacionada con el éxito académico en el ramo de introducción a la programación en la Universidad Andrés Bello. Se investiga si la guía, que consta de 52 ejercicios y no es obligatoria, ayuda a los estudiantes a enfrentar la primera prueba del ramo y si puede predecir la deserción en la carrera. 
Además, se propone el uso del marco unificado SHAP y XAI para analizar y comprender cómo las guías de programación influyen en el rendimiento de los estudiantes. El estudio incluye una revisión bibliográfica sobre estudios previos relacionados, un análisis descriptivo de los datos recolectados y una discusión sobre las implicaciones prácticas y teóricas del estudio. 
Los resultados obtenidos serán útiles para mejorar las estrategias de apoyo a los estudiantes y la toma de decisiones relacionadas con el uso de las guías como herramienta educativa.
\vfill