\begin{center}
    \textbf{\LARGE \textit{Análisis de la relación entre la resolución de una guía de programación y el éxito académico en el ramo de introducción a la programación}}
    
    \vspace{1cm} % Espacio vertical para separar el título del resto

    \fontsize{12}{14}\selectfont
    Gastón Ernesto Sepulveda Espinoza \\

    % \vspace{1cm} % Espacio vertical para separar el título del resto

    % \fontsize{12}{14}\selectfont
    Bajo la supervisión del Profesor Billy Mark Peralta Márquez en la Universidad Andrés Bello

\end{center}

\vspace{1cm} % Espacio vertical para separar el título del resto

% \newpage

% \begin{center}
%     \textbf{\LARGE \textit{Abstract}}
% \end{center}

% This study delves into the potential causal relationship between the completion of a non-mandatory programming guide, consisting of 52 exercises, and academic success in the introductory programming course at Universidad Andrés Bello. It evaluates the guide's role in preparing students for the course's initial test and its predictive capacity for program dropout rates. Utilizing the unified SHAP and XAI framework, the research seeks to understand the causal influences of programming guides on student performance. The paper encompasses a comprehensive literature review, a detailed analysis of the gathered data, and a discussion on the study's practical and theoretical implications, emphasizing causal inference. The findings aim to enhance student support strategies and inform decisions regarding the educational utility of such guides.


\begin{center}
    \textbf{\LARGE \textit{Resumen}}
\end{center}

\fontsize{12}{14}\selectfont
Este estudio profundiza en la posible relación causal entre la finalización de una guía de programación no obligatoria, compuesta por 52 ejercicios, y el éxito académico en el curso introductorio de programación en la Universidad Andrés Bello. Evalúa el papel de la guía en la preparación de los estudiantes para la primera prueba del curso y su capacidad predictiva para las tasas de deserción del programa. Utilizando el marco unificado SHAP y XAI, la investigación busca entender las influencias causales de las guías de programación en el rendimiento de los estudiantes. El documento abarca una revisión literaria exhaustiva, un análisis detallado de los datos recopilados y una discusión sobre las ramificaciones prácticas y teóricas del estudio, enfatizando la inferencia causal. Los hallazgos buscan mejorar las estrategias de apoyo estudiantil e informar decisiones respecto a la utilidad educativa de tales guías.
