\begin{center}
    \textbf{\LARGE \textit{Título}}
    \vfill
    Gastón Ernesto Sepulveda Espinoza
    \vfill
    Bajo la supervisión del Profesor Billy Mark Peralta Márquez en la Universidad Andrés Bello
    \vfill
\end{center}
\vfill
\begin{center}
    \textbf{\LARGE \textit{Resumen}}
\end{center}
\vfill
En este trabajo se examina si la resolución de una guía de programación 
está relacionada con el éxito académico en el ramo de introducción a la 
programación en la Universidad Andrés Bello. Se investiga si la guía, que 
consta de 52 ejercicios y no es obligatoria, ayuda a los estudiantes 
a enfrentar la primera prueba del ramo y si puede predecir la deserción 
en la carrera. 
Además, se propone el uso de redes bayesianas para generar predicciones 
sobre su relevancia para aprobar el ramo y su capacidad predictiva para 
la deserción. El estudio incluye una revisión bibliográfica sobre estudios 
previos relacionados, un análisis descriptivo de los datos recolectados y 
una discusión sobre las implicaciones prácticas y teóricas del estudio. 
Los resultados sugieren que resolver la guía puede estar relacionado con 
el éxito académico y pueden ser útiles para predecir la deserción en el ramo.
\vfill