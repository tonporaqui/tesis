\hypertarget{propuesta}{%
    \section{Propuesta}\label{propuesta}}

    \vfill
Con base en el problema planteado, se propone llevar a cabo una investigación detallada para analizar la relación entre la resolución de una guía de programación y el éxito académico en el ramo de introducción a la programación de la Universidad Andrés Bello. La propuesta de investigación se estructura de la siguiente manera:
\vfill
\begin{enumerate}
    \item Recopilación de datos: Se obtendrá el conjunto de datos "dataset a 2021", que contiene información sobre los resultados de la guía y el rendimiento en la solemne 1 del año 2021. Además, se recopilarán datos demográficos de los estudiantes, como género, edad y antecedentes académicos relevantes.

    \item Revisión bibliográfica: Se realizará una investigación exhaustiva de la literatura académica de los últimos 10 años relacionada con el tema. Se analizarán estudios previos que aborden la relación entre la resolución de guías de programación y el éxito académico, así como investigaciones sobre el uso de redes bayesianas para la predicción en contextos educativos.

    \item Análisis descriptivo: Se llevará a cabo un análisis descriptivo de los datos recopilados, incluyendo medidas de tendencia central y dispersión, para examinar la distribución de los resultados de la guía y la solemne 1. Se realizarán comparaciones entre grupos de estudiantes para identificar posibles correlaciones.

    \item Construcción del modelo de red bayesiana: Utilizando los datos recopilados, se desarrollará un modelo de red bayesiana que permita evaluar la influencia de la resolución de la guía en el éxito académico y su capacidad predictiva para la deserción en la carrera. Se definirán las variables relevantes y se establecerán las relaciones probabilísticas entre ellas.

    \item Validación del modelo: Se realizarán pruebas y validaciones del modelo de red bayesiana utilizando técnicas como validación cruzada y análisis de sensibilidad. Esto garantizará la confiabilidad y robustez de las predicciones generadas por el modelo.

    \item Análisis e interpretación de resultados: Se realizará un análisis de los resultados obtenidos, considerando la influencia de la guía de programación en el éxito académico y su capacidad predictiva de deserción. Se examinarán las relaciones identificadas por el modelo de red bayesiana y se evaluará su significancia estadística.

    \item Conclusiones y recomendaciones: Se presentarán las conclusiones derivadas de la investigación, destacando las principales contribuciones y hallazgos. Además, se ofrecerán recomendaciones prácticas para la universidad en cuanto al uso de la guía de programación como recurso educativo y su impacto en el rendimiento académico de los estudiantes.
\end{enumerate}
\vfill
La realización de esta propuesta de investigación permitirá obtener una comprensión más profunda de la relación entre la resolución de una guía de programación y el éxito académico en el ramo de introducción a la programación. Además, proporcionará información valiosa para mejorar las estrategias de apoyo a los estudiantes y tomar decisiones informadas en relación con la implementación y promoción de la guía como recurso educativo en la universidad.
\vfill