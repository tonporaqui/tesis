\tableofcontents

\hypertarget{contexto}{%
    \section{Contexto}\label{contexto}}

Este estudio examina si la resolución de una guía de programación está relacionada con el éxito
académico en el ramo de introducción a la programación de la Universidad Andrés Bello. Se investiga
si la guía, que consta de 52 ejercicios y no es obligatoria, ayuda a los estudiantes a enfrentar la
primera prueba del ramo. Además, se analiza si la guía puede predecir la deserción en la carrera.

Se realiza una revisión bibliográfica de los últimos 10 años para recopilar estudios relacionados.
Se propone el uso de redes bayesianas para generar predicciones sobre la relevancia de la guía
en la aprobación del ramo y su capacidad predictiva para la deserción.

Los resultados contribuirán a comprender mejor la influencia de la guía de programación
en el éxito académico de los estudiantes. Además, se espera que los hallazgos mejoren
las estrategias de apoyo a los estudiantes y la toma de decisiones de la universidad
en relación con la guía como recurso educativo.

Palabras clave: guía de programación, éxito académico, redes bayesianas, predicción, deserción, introducción a la programación.

\hypertarget{problema}{%
    \section{Problema}\label{problema}}

El problema de esta investigación se centra en determinar la relación entre
la resolución de una guía de programación y el éxito académico en el ramo
de introducción a la programación de la Universidad Andrés Bello. Se busca responder
a las siguientes preguntas: ¿La resolución de la guía es significativa para aprobar el ramo?
¿La guía puede ser una herramienta predictiva de deserción en la carrera?

Además, se plantea la necesidad de analizar la relevancia de la guía en la preparación de
los estudiantes para la primera prueba o solemne 1 del ramo, evaluando si la resolución de
los ejercicios de la guía proporciona una base sólida para afrontar exitosamente la evaluación.

Para abordar este problema, se utilizará un conjunto de datos llamado "dataset a 2021",
que contiene información sobre los resultados de la guía y el rendimiento en la
solemne 1 del año 2021. Se realizará una revisión bibliográfica de los últimos 10 años p
ara recopilar información relevante y se aplicará el enfoque predictivo de redes bayesianas
para analizar la relación entre la resolución de la guía y el éxito académico,
así como su posible utilidad como predictor de deserción.

El estudio de este problema tiene como objetivo proporcionar una comprensión más precisa
sobre la importancia de la guía de programación y su influencia en el rendimiento académico de los estudiantes.
Los resultados obtenidos permitirán mejorar las estrategias de apoyo a los estudiantes y la toma de decisiones
por parte de la universidad en relación con la implementación y promoción de la guía como recurso educativo
en el ramo de introducción a la programación.

\hypertarget{discusiuxf3n}{%
    \section{Discusión}\label{discusiuxf3n}}

La aplicación de modelos gráficos y el enfoque bayesiano en el análisis de datos ha demostrado ser una herramienta poderosa
en la modelización de sistemas complejos (Koller \& Friedman, 2009 \cite{koller2009introduction}). Estos modelos proporcionan
una forma intuitiva de representar las relaciones entre variables y permiten el razonamiento probabilístico mediante el uso de reglas de inferencia bayesianas.

En el campo del análisis de datos, el uso de técnicas avanzadas ha permitido extraer información valiosa y conocimiento profundo de conjuntos
de datos complejos (Hair et al., 2019 \cite{hair2019advanced}). Estas técnicas incluyen métodos estadísticos y algoritmos de aprendizaje
automático que ayudan a descubrir patrones, identificar relaciones causales y realizar predicciones precisas.

En el razonamiento de redes bayesianas con evidencia incierta, se busca manejar la incertidumbre presente en los datos observados y
actualizar las creencias sobre las variables de interés (Jensen, 2001 \cite{jensen2001bayesian}). Las redes bayesianas proporcionan
una forma de representar y actualizar el conocimiento incierto mediante la propagación de creencias utilizando reglas de inferencia probabilística.

El análisis de datos ha evolucionado desde la minería de datos hacia enfoques más sofisticados y avanzados (Han et al., 2011 \cite{han2011data}).
Ahora se enfoca en explorar relaciones causales y descubrir conocimiento más profundo a partir de los datos.

La inferencia causal utilizando redes bayesianas y el cálculo do ha surgido como una herramienta poderosa para comprender las relaciones
causales en conjuntos de datos observacionales (Pearl, 2009 \cite{pearl2009introduction}). Estos modelos permiten identificar las relaciones
causales subyacentes y realizar inferencias causales a partir de datos observacionales.

La predicción del abandono escolar es un tema de investigación activo y de gran relevancia en el campo educativo (Wang et al., 2017 \cite{wang2017literature}).
La aplicación de técnicas de minería de datos ha demostrado ser útil para identificar a los estudiantes en riesgo y proporcionar intervenciones tempranas.

Es fundamental tener modelos interpretables para comprender y confiar en las predicciones generadas por los algoritmos de aprendizaje automático
(Kocev et al., 2013 \cite{kocev2013need}). Estos modelos permiten a los responsables de la toma de decisiones comprender los resultados y tomar
acciones adecuadas.

La predicción del rendimiento académico mediante técnicas de minería de datos ha demostrado ser un enfoque prometedor
(García et al., 2018 \cite{garcia2018prediccion}). La aplicación de estas técnicas permite identificar patrones y tendencias en los datos académicos,
lo que puede ayudar a predecir el rendimiento futuro de los estudiantes.

En resumen, el uso de modelos gráficos, el enfoque bayesiano, técnicas avanzadas de análisis de datos y minería de datos ofrece una perspectiva integral para
comprender y predecir el rendimiento académico y el abandono escolar (Koller \& Friedman, 2009 \cite{koller2009introduction};
Hair et al., 2019 \cite{hair2019advanced}; Jensen, 2001 \cite{jensen2001bayesian}; Han et al., 2011 \cite{han2011data}; Pearl, 2009 \cite{pearl2009introduction};
Wang et al., 2017 \cite{wang2017literature}; Kocev et al., 2013 \cite{kocev2013need}; García et al., 2018 \cite{garcia2018prediccion}).
Estas herramientas proporcionan una base sólida para la toma de decisiones informadas en el ámbito educativo y la implementación de intervenciones tempranas
para mejorar el éxito estudiantil.

Los modelos gráficos y las redes bayesianas permiten capturar las complejas relaciones entre las variables académicas. Además, 
estas técnicas proporcionan un marco flexible y probabilístico para modelar el rendimiento académico, realizar inferencias causales y 
analizar la incertidumbre presente en los datos \cite{jensen2001bayesian}.

La aplicación de técnicas avanzadas de análisis de datos y minería de datos, como algoritmos de aprendizaje automático, ha permitido descubrir patrones ocultos
y relaciones causales en conjuntos de datos complejos \cite{hair2019advanced, han2011data}. Estas técnicas ayudan a identificar factores predictivos y
proporcionan modelos de predicción precisos para predecir el rendimiento académico y el abandono escolar.

Es importante destacar la necesidad de modelos interpretables que permitan comprender y confiar en las predicciones generadas \cite{kocev2013need}.
La interpretabilidad es crucial en el contexto educativo, ya que los responsables de la toma de decisiones deben comprender los factores que influyen
en el rendimiento académico y tomar acciones adecuadas para mejorar los resultados de los estudiantes.

La predicción del rendimiento académico mediante técnicas de minería de datos ofrece un enfoque prometedor para identificar a los estudiantes en riesgo
y proporcionar intervenciones tempranas \cite{garcia2018prediccion, wang2017literature}. Al analizar datos académicos históricos y otros factores relevantes,
se pueden desarrollar modelos predictivos que ayuden a identificar a los estudiantes que pueden necesitar apoyo adicional.

En conclusión, la combinación de modelos gráficos, enfoques bayesianos, técnicas avanzadas de análisis de datos y minería de datos ofrece un enfoque poderoso
para comprender y predecir el rendimiento académico y el abandono escolar. Estas herramientas proporcionan a educadores y responsables de políticas una base
sólida para la toma de decisiones informadas y la implementación de estrategias efectivas para mejorar el éxito estudiantil.



\hypertarget{propuesta}{%
    \section{Propuesta}\label{propuesta}}


Con base en el problema planteado, se propone llevar a cabo una investigación detallada para analizar la relación entre la resolución de una guía de programación y el éxito académico en el ramo de introducción a la programación de la Universidad Andrés Bello. La propuesta de investigación se estructura de la siguiente manera:

\begin{enumerate}
    \item Recopilación de datos: Se obtendrá el conjunto de datos "dataset a 2021", que contiene información sobre los resultados de la guía y el rendimiento en la solemne 1 del año 2021. Además, se recopilarán datos demográficos de los estudiantes, como género, edad y antecedentes académicos relevantes.

    \item Revisión bibliográfica: Se realizará una investigación exhaustiva de la literatura académica de los últimos 10 años relacionada con el tema. Se analizarán estudios previos que aborden la relación entre la resolución de guías de programación y el éxito académico, así como investigaciones sobre el uso de redes bayesianas para la predicción en contextos educativos.

    \item Análisis descriptivo: Se llevará a cabo un análisis descriptivo de los datos recopilados, incluyendo medidas de tendencia central y dispersión, para examinar la distribución de los resultados de la guía y la solemne 1. Se realizarán comparaciones entre grupos de estudiantes para identificar posibles correlaciones.

    \item Construcción del modelo de red bayesiana: Utilizando los datos recopilados, se desarrollará un modelo de red bayesiana que permita evaluar la influencia de la resolución de la guía en el éxito académico y su capacidad predictiva para la deserción en la carrera. Se definirán las variables relevantes y se establecerán las relaciones probabilísticas entre ellas.

    \item Validación del modelo: Se realizarán pruebas y validaciones del modelo de red bayesiana utilizando técnicas como validación cruzada y análisis de sensibilidad. Esto garantizará la confiabilidad y robustez de las predicciones generadas por el modelo.

    \item Análisis e interpretación de resultados: Se realizará un análisis de los resultados obtenidos, considerando la influencia de la guía de programación en el éxito académico y su capacidad predictiva de deserción. Se examinarán las relaciones identificadas por el modelo de red bayesiana y se evaluará su significancia estadística.

    \item Conclusiones y recomendaciones: Se presentarán las conclusiones derivadas de la investigación, destacando las principales contribuciones y hallazgos. Además, se ofrecerán recomendaciones prácticas para la universidad en cuanto al uso de la guía de programación como recurso educativo y su impacto en el rendimiento académico de los estudiantes.
\end{enumerate}

La realización de esta propuesta de investigación permitirá obtener una comprensión más profunda de la relación entre la resolución de una guía de programación y el éxito académico en el ramo de introducción a la programación. Además, proporcionará información valiosa para mejorar las estrategias de apoyo a los estudiantes y tomar decisiones informadas en relación con la implementación y promoción de la guía como recurso educativo en la universidad.